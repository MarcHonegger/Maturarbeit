\chapter{Team Management}

\section{Anfänge}
\begin{itemize}
    \item Bereits Mitte Januar 2022 fragte Marc Elia an, ob sie nicht zusammen die Maturitätsarbeit schreiben wollten. Wir hatten beide vor, als Maturaarbeit etwas zu programmieren. Also vereinigten wir unsere Kräfte.
    \item Die nächste Frage war, was wir genau als Projekt programmieren wollten. Es stellte sich sehr schnell heraus, dass sich Marcs Grundidee (ein Videospiel) perfekt eignen würde. Man kann seiner Fantasie bei der Erstellung eines Videospiels
    freien Lauf lassen.
    \item Marc brachte am Anfang direkt den Vorschlag, dass wir unsere Zeit mit Jira tracken würden. Er setzte zudem GitHub und Unity für unseren ersten Gebrauch auf. 
    \item Elia konnte sich sehr schnell in dieser unbekannten Umgebung zurechtfinden. So entstand ein begeisterungsfähiges gleichberechtigtes Duo.
\end{itemize}

\section{Betreuungsperson}
\begin{itemize}
    \item Uns wurde vor Beginn der Arbeit mehrfach gesagt, dass einige Schüler Probleme haben, eine Betreuungsperson zu finden. Um sicherzugehen, dass dies uns nicht geschehen würde, suchten wir bereits sehr früh nach einer Lehrperson.
    \item Elias erster Vorschlag war Herr Albert Kern. Er hat ihn bereits vor längerer Zeit provisorisch für eine individuelle Betreuung angefragt. Was lag näher, als ihn nun auch für eine Gruppenarbeit anzufragen? Wie es der Zufall wollte, begegneten sich 
    Herr Kern und Elia Ende Februar zufälligerweise auf dem Gang und tauschten einige Worte aus. Herr Kern gab allerdings zu, dass er keine Erfahrung mit der Entwicklung von Videospielen hatte und schlug Herrn Martin Hunziker vor.
    Dieser sei ein begnadeter Videospiel-Fanatiker und wahrscheinlich besser für die Betreuung geeignet. 
    \item Am dritten März schrieben wir per E-Mail eine kurze Anfrage an Herrn Hunziker. 
    Bereits 4 Tage später sassen wir mit Herrn Hunziker in seinem Büro und besprachen mit ihm unsere Ideen. Er sagt uns, er habe selbst bisher keine Erfahrung in der Programmierung von Videospielen.
    \item Eine Woche später gab er sein schriftliches Einverständnis für die Betreuung unserer Maturitätsarbeit. 
    \item Wir informierten ihn regelmässig über unseren Fortschritt und konnten uns bei Fragen an ihn wenden.
\end{itemize}

\section{Kommunikation}
\begin{itemize}
    \item Die Kommunikation zwischen Elia und Marc fand vor allem über WhatsApp statt. Dies geschah allerdings nicht nur in Textform, sondern des Öfteren auch in Telefonaten zu allen möglichen, aber auch unmöglichen Uhrzeiten. Zudem konnten wir auf Jira nachschauen, auf welche Aufgabe der Partner
    seine Zeit getracked hat. 
    \item Zudem trafen wir uns dreimal physisch ausserhalb der Schulzeit, um unsere Arbeit zu organisieren.
    \item Wir müssen allerdings zugeben, dass die Kommunikation teilweise nicht ausreichend vorhanden war. Manchmal hatten wir keine Ahnung, woran genau der Partner arbeitete. Wir wussten zwar den ungefähren Arbeitsinhalt, allerdings nicht immer alle Einzelheiten.
    \item Die Kommunikation zwischen Schüler und Lehrer war unserer Meinung nach mit sechs Treffen genügend gut.
\end{itemize}



