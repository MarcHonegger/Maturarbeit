\chapter{Team Management}

\section{Anfänge}
\begin{itemize}
    \item Bereits Mitte Januar fragte Marc Elia an, ob sie nicht zusammen die Maturitätsarbeit schreiben wollten. Wir beide hatten vor, etwas zu programmieren als Maturaarbeit. Da ergab sich die Idee, dass man die 
    Kräfte vereinigte und zusammen ein Projekt programmiert. 
    \item Die nächste Frage war, was wir als Projekt genau programmieren wollten. Es stellte sich sehr schnell heraus, dass sich ein Videospiel perfekt eignen würde. Dies, da man sich bei einem Videospiel unter anderem frei 
    ausleben kann. Als feststand, dass wir ein Team bilden würden, mussten wir noch eine Lehrperson finden, die uns betreuen würde.
    \item Marc brachte am Anfang direkt den Vorschlag, dass wir unsere Zeit mit Jira tracken würden. Er setzte zudem Github und Unity für unseren ersten Gebraucht auf. 
\end{itemize}

\section{Betreuungsperson}
\begin{itemize}
    \item Uns wurde vor Beginn der Arbeit mehrfach gesagt, dass einige Schüler Probleme haben, eine Betreuungsperson zu finden. Um sicherzugehen, dass dies uns nicht geschehen würde, suchten wir bereits sehr früh nach einer Lehrperson.
    \item Der erste Vorschlag war Albert Kern. Elia hatte ihn am Anfang für eine individuelle Arbeit angefragt. Was lag näher, nun ihn auch für eine Gruppenarbeit anzufragen. Wie es der Zufall wollte, begegneten sich 
    Herr Kern und Elia Ende Februar zufälligerweise auf dem Gang und tauschten einige Worte aus. Herr Kern gab allerdings zu, dass er keine Erfahrung mit der Entwicklung von Videospielen hatte und schlug Elia Martin Hunziker vor.
    Dieser sei ein begnadeter Videospiel-Fanatiker und wahrscheinlich besser für die Betreuung geeignet. 
    \item Am dritten März schrieben wir eine kurze Anfrage an Herr Hunziker per E-Mail. Herr Kern hatte zuvor im Gespräch mit Elia erwähnt, dass er dabei gerne ein gutes Wort bei Herr Hunziker einlegen würde. Ob dies tatsächlich 
    stattfand, entzieht sich unserem Wissen. Allerdings sassen wir 4 Tage später, am siebten März bereits bei Herr Hunziker im Büro und besprachen mit ihm unsere Idee. 
    \item Eine Woche später stand seine Unterschrift auf unseren beiden Blättern bezüglich der Maturitätsarbeit. Allerdings hat selbst Herr Hunziker gestanden, dass er uns bei den meisten Problemen nicht helfen können, weil er
    keine grosse Erfahrung hat.
\end{itemize}

\section{Kommunikation}
\begin{itemize}
    \item Unsere Kommunikation fand vor allem über WhatsApp statt. Dies geschah allerdings nicht nur in Textform, sondern des Öfteren auch in Telefonaten. Zudem konnte man auf Jira nachschauen, auf welche Aufgabe der Partner
    seine Zeit getracked hat. 
    \item Wir müssen allerdings zugeben, dass die Kommunikation teilweise sehr gering vorhanden war. Teilweise hatte man keine Ahnung, woran genau der Partner arbeitete. Wir wussten zwar das Ungefähre, allerdings manchmal keine Einzelheiten.
    \item Die Kommunikation zwischen Schüler und Lehrer war unserer Meinung nach mit sechs Treffen genügend gut. Zudem trafen wir uns dreimal neben der Schule, um unsere Arbeit zu organisieren.
\end{itemize}



