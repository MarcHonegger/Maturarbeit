\chapter{Team Management}

\section{Anfänge}
\begin{itemize}
    \item Bereits Mitte Januar 2022 fragte Marc Elia an, ob sie nicht zusammen die Maturitätsarbeit schreiben wollten. Wir hatten beide vor, als Maturaarbeit etwas zu programmieren. Also vereinigten wird unsere Kräfte.
    \item Die nächste Frage war, was wir genau als Projekt programmieren wollten. Es stellte sich sehr schnell heraus, dass sich Marcs Grundidee (ein Videospiel), perfekt eignen würde. Man kann seiner Fantasie bei der Erstellung eines Videospiels
    freien Lauf lassen.
    \item Marc brachte am Anfang direkt den Vorschlag, dass wir unsere Zeit mit Jira tracken würden. Er setzte zudem Github und Unity für unseren ersten Gebrauch auf. 
    \item Elia konnte sich sehr schnell in dieser unbekannten Umgebung zurechtfinden. So entstand ein begeisterungsfähiges gleichberechtigtes Duo.
\end{itemize}

\section{Betreuungsperson}
\begin{itemize}
    \item Uns wurde vor Beginn der Arbeit mehrfach gesagt, dass einige Schüler Probleme haben, eine Betreuungsperson zu finden. Um sicherzugehen, dass dies uns nicht geschehen würde, suchten wir bereits sehr früh nach einer Lehrperson.
    \item Elias erster Vorschlag war Herr Albert Kern. Er hat ihn bereits vor längerer Zeit provisorisch für eine individuelle Betreuung angefragt. Was lag näher, als ihn nun auch für eine Gruppenarbeit anzufragen? Wie es der Zufall wollte, begegneten sich 
    Herr Kern und Elia Ende Februar zufälligerweise auf dem Gang und tauschten einige Worte aus. Herr Kern gab allerdings zu, dass er keine Erfahrung mit der Entwicklung von Videospielen hatte und schlug Elia Martin Hunziker vor.
    Dieser sei ein begnadeter Videospiel-Fanatiker und wahrscheinlich besser für die Betreuung geeignet. 
    \item Am dritten März schrieben wir eine kurze Anfrage an Herr Hunziker per E-Mail. Herr Kern hatte zuvor im Gespräch mit Elia erwähnt, dass er dabei gerne ein gutes Wort bei Herr Hunziker einlegen würde. Ob dies tatsächlich 
    stattfand, entzieht sich unserem Wissen. Allerdings sassen wir 4 Tage später, am siebten März bereits bei Herr Hunziker im Büro und besprachen mit ihm unsere Idee. 
    \item Eine Woche später stand seine Unterschrift auf unseren beiden Blättern bezüglich der Maturitätsarbeit. Allerdings hat selbst Herr Hunziker gestanden, dass er uns bei den meisten Problemen nicht helfen können, weil er
    keine grosse Erfahrung hat.
\end{itemize}

\section{Kommunikation}
\begin{itemize}
    \item Unsere Kommunikation fand vor allem über WhatsApp statt. Dies geschah allerdings nicht nur in Textform, sondern des Öfteren auch in Telefonaten. Zudem konnte man auf Jira nachschauen, auf welche Aufgabe der Partner
    seine Zeit getracked hat. 
    \item Wir müssen allerdings zugeben, dass die Kommunikation teilweise sehr gering vorhanden war. Teilweise hatte man keine Ahnung, woran genau der Partner arbeitete. Wir wussten zwar das Ungefähre, allerdings manchmal keine Einzelheiten.
    \item Die Kommunikation zwischen Schüler und Lehrer war unserer Meinung nach mit sechs Treffen genügend gut. Zudem trafen wir uns dreimal neben der Schule, um unsere Arbeit zu organisieren.
\end{itemize}



