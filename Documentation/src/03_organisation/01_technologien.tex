\chapter{Technologien}

\section{Unity}
\url{https://unity.com/}\\
''Unity ist mehr als eine Engine'', schreibt Unity auf der eigenen Webseite \url{https://unity.com/pages/more-than-an-engine}. Fakt ist, Unity ist die marktführende Game-Engine. Die Laufzeit- und Entwicklungsumgebung 
ist vor allem bei \gls{Indie Developer}n sehr beliebt. Deswegen ist es auch kein Wunder, dass über 50 Prozent aller Spiele mit Unity erstellt wurden. 
Doch auch für Grafikdesigner, Architekten, Filmentwickler und viele Weitere ist Unity ein sehr mächtiges Programm. Und wenn selbst die Streitkräfte der Vereinigten Staaten Unity nutzen, wieso sollten wir 
noch weiter suchen? Trotzdem haben wir einige für uns wichtige Vorteile herausgesucht:\\

\begin{itemize}
    \item Physik
    \begin{itemize}
        \item Unity berechnet einen grossen Teil der physikalischen Eigenschaften automatisch. Dies ist für uns von grossem Vorteil, da wir uns zum Glück nicht mit dem Berechnen von Kollisionen herumschlagen müssen.
        Ausserdem erlaubt uns Unity die Physik nach Bestreben zu ändern und ganz nach unserem Willen zu formen.
    \end{itemize}
    \item Grafiken
    \begin{itemize}
        \item Die Animation von Grafiken ist dank Unity ziemlich einfach. So kann man beispielsweise eine Bildersequenz in Unity laden, und Unity formt daraus automatisch eine Animation. Jedes Spielobjekt kann seinen
        eigenen Animator haben. In diesem kann man verschieden Zuständen des Spielobjekts verschiedene Animationen zuweisen. Greift die Truppe zum Beispiel an, wird von der normalen Animation zur Angriffsanimation
        gewechselt.
    \end{itemize}
    \item Programmierung
    \begin{itemize}
        \item In Unity kann man Spielobjekten selbst geschriebene Programme, sogenannte Scripts, hinzufügen. Diese werden mit C\# geschrieben. Dies war von Vorteil, da wir bereits Erfahrung mit C\# hatten. 
    \end{itemize}
\end{itemize}

\section{GitHub}
\url{https://github.com/}\\
GitHub ist die grösste webbasierte open-source Plattform. Auf ihr kann man Source Code speichern und teilen. Ein grosser Vorteil von GitHub ist, dass man immer zu vergangenen Versionen zurückgelangen kann.
Hinzu kommt, dass man verschiedene Versionen des Source Codes in sogenannten Branches speichern und verändern kann. GitHub speichert ausserdem jede Änderung am Code, die durchgeführt wurde. So entsteht eine History, die jede Änderung nachverfolgen lässt. 
Ein sehr grosser Vorteil von GitHub ist, dass man im Team in Echtzeit am gleichen Code schreiben kann. Unser grösstes Pro-Argument: Backup-Backup-Backup!
Alles wird gespeichert, und es gibt keine Gründe, sich die Festplatte mit Backups vollzuschreiben. \\
Eine Visualisierung von unserem \gls{GitHub Repository} kann man hier finden: \url{https://www.youtube.com/watch?v=IQT37uwpcg4}


\section{Jetbrains Rider}
\url{https://www.jetbrains.com/rider/}\\
''Rider ist eine unglaubliche .NET-IDE mit der Power von ReSharper'', schreibt Jetbrains auf der eigenen Webseite. Dies können wir nur bestätigen. Rider hat zudem die Unity API integriert. Dadurch hat Rider eine
spezialisierte Codeinspektion für Unity. Treten Errors im Unity Editor auf, kann Rider direkt den Ursprung des Fehlers anzeigen. Das Debugging ist dadurch um einiges leichter und müheloser. 
Rider warnt User ausserdem vor schlecht geschriebenen Funktionen, die einen Performance-kritischen Effekt auf das Spiel haben könnten. 


\section{Jira}
\url{https://www.atlassian.com/software/jira}\\
Jira ist ein webbasiertes Programm für Projektmanagement und Zeiterfassung. Dies ermöglichte uns genau zu erfassen, wie viel Zeit wir an einem bestimmten Problem gearbeitet haben. Uns war klar, dass wir eine Art 
der Zeiterfassung brauchten, denn im Nachhinein ist dies ein Ding der Unmöglichkeit.

\section{LaTeX}
\url{https://www.latex-project.org/}\\
LaTeX verwendet das Textsatzsystem TeX. LaTeX ist eine Software, die sich von anderen Textprogrammen unterscheidet. Im Gegensatz zu den gewohnten WYSIWYG (''What You See is what you get'') Editoren (wie z.B. Word)
braucht man zur Benutzung von LaTeX einen separaten Editor. LaTeX vereinfacht den Umgang mit TeX durch einfachere Befehle. Der grösste Vorteil an TeX ist, dass man sehr genau einstellen kann, wie alles aussehen soll.


\section{Stable Diffusion}
\url{https://stability.ai/}\\
Stable-Diffusion ist ein State-of-the-Art, Deep-Learning Text-zu-Bild Modell. Es wurde 2022 vom Start-up Stability AI veröffentlicht. Es wird für die Generation von detaillierten Bildern verwendet. Wir wollten seit
Beginn der Arbeit eine sehr neue Technologie verwenden. Dann als Stable Diffusion veröffentlicht wurde, war uns klar, dass wir diese Chance nicht verpassen dürfen. Hier ergab sich, dass wir Stable Diffusion verwendet haben, um einen individuellen
Hintergrund und ein individuelles Icon zu generieren. Wir nutzten Ideen, eine schlechte Skizze und viele Stunden an Testen und Verbessern, um den Hintergrund und das Icon nach unserem Geschmack zu generieren.

\section{Visual Studio Code}
\url{https://code.visualstudio.com/}\\
Visual Studio Code ist ein kostenloser Editor von Microsoft. Ein grosser Vorteil von Visual Studio Code ist, dass man Erweiterungen hinzufügen kann. Die Erweiterung ''LaTeX Workshop'' (\url{https://marketplace.visualstudio.com/items?itemName=James-Yu.latex-workshop})
hat dabei das spezifische Syntax-highlighting hinzugefügt. Ohne jene Erweiterung sich die Benutzung von LaTeX um einiges schwerer gewesen.

\section{Gource}
\url{https://gource.io/}\\
Gource ist eine gratis open-source Software, die perfekt für die Visualisierung von GitHub Repositories geeignet ist. Wir haben Gource für die Erstellung eines Videos genutzt, das als Visualisierung unseres Dateisystems
dient (\url{https://www.youtube.com/watch?v=IQT37uwpcg4}).

