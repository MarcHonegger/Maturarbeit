\chapter{Technologien}

\section{Unity}
\url{https://unity.com/}\\
''Unity ist mehr als eine Engine'', schreibt Unity auf der eigenen Webseite \url{https://unity.com/pages/more-than-an-engine}. Fakt ist, Unity ist marktführend. Die Laufzeit- und Entwicklungsumgebung 
ist vor allem bei Indie-Developern sehr beliebt. Deswegen ist es auch kein Wunder, dass über 50 Prozent aller Spiele mit Unity erstellt wurden. 
Doch auch für Grafikdesigner, Architekten, Filmentwickler und viele weitere ist Unity ein sehr mächtiges Programm. Und wenn selbst die Streitkräfte der Vereinigten Staaten Unity nutzen, wieso sollten wir 
noch weiter suchen? Trotzdem haben wir einige für uns wichtige Vorteile herausgesucht:\\

\begin{itemize}
    \item Physik
    \begin{itemize}
        \item Unity berechnet einen grossen Teil der physikalischen Eigenschaften automatisch. Dies ist für uns sehr vom Vorteil, da wir uns zum Glück nicht mit Kollisionen herumschlagen müssen.
        Ausserdem erlaubt uns Unity die Physik nach Bestreben zu ändern und ganz nach unserem Willen zu formen.
    \end{itemize}
    \item Grafiken
    \begin{itemize}
        \item Die Animation von Grafiken ist dank Unity ziemlich einfach. So kann man zum Beispiel eine Bildersequenz in Unity laden, und Unity formt automatisch eine Animation daraus. Jedes Spielobjekt kann seinen
        eigenen Animator haben. In diesem kann man dann verschieden Zuständen des Spielobjekts verschiedene Animationen zuweisen. Greift die Truppe zum Beispiel an, wird von der normalen Animation zur Angriffs-Animation
        gewechselt.  
    \end{itemize}
    \item Programmierung
    \begin{itemize}
        \item In Unity kann man Spielobjekten selbst geschrieben Programme, sogenannte Scripts, hinzufügen. Diese werden mit C\# geschrieben. Dies ist von Vorteil, da wir bereits Erfahrung mit C\# haben. 
    \end{itemize}
\end{itemize}

\section{Github}
\url{https://github.com/}

\section{Jetbrains Rider}
\url{https://www.jetbrains.com/rider/}\\
''Rider ist eine unglaubliche .NET-IDE mit der Power von ReSharper'', schreibt Jetbrains auf der eigenen Webseite. Dies können wir nur bestätigen. Rider hat zudem die Unity API integriert. Dadurch hat Rider eine
spezialisierte Codeinspektion für Unity. Treten Errors im Unity Editor auf, kann Rider direkt den Ursprung des Fehlers anzeigen. Das Debugging ist dadurch um einiges leichter und müheloser. 
Rider warnt User ausserdem vor Performance-kritischen Funktionen. 


\section{Jira}
\url{https://www.atlassian.com/software/jira}\\
Webbasiertes Programm für Projektmanagement und Zeiterfassung.


\section{LaTex}
\url{https://www.latex-project.org/}\\

\section{Stable-Diffusion}
\url{https://stability.ai/}\\

