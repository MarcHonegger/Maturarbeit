\chapter{Team}

\section{Rückblick}

\subsection*{Teamarbeit}
Wir verstanden uns während der gesamten Zeit der Arbeit sehr gut. Es kam nie zu zwischenmenschlichen Problemen. Jedoch haben wir viel Potenzial der Teamarbeit nicht ausgeschöpft.
Da wir kein gemeinsames Code Ownership hatten, gibt es nun Softwareteile, die nur einer von uns detailliert kennt und versteht. Diese Spezialisierung führte zu einem teilweise getrennten Arbeitsprozess. Zurückführen kann man dies auch auf die zu seltenen
Treffen. Hätten wir uns öfter als dreimal getroffen, wären viele Unklarheiten oder Missverständnisse nicht aufgetreten. So waren wir beide des Öfteren unsicher, wie genau wir unsere verrichtete Arbeit
dokumentieren sollten. Es stellte sich zudem die Frage, was genau als investierte Zeit geltet.
Unsere grössten Fehler waren, dass wir nie Pull Requests und in die letzten zwei Monate nicht mehr mit Feature Branches gearbeitet haben. Die führte zu verminderter Code-Qualität
und entsprechend grossem Zeitverlust bei der Fehlersuche.


\subsection*{Projektplanung}
Ein Ziel vor den Augen zu haben und eine Vision zu verfolgen ist wichtig. Jedoch soll es
immer möglich sein, diese anzupassen. Bereits im März haben wir sehr viel getüftelt und Ideen entwickelt, die zu diesem Zeitpunkt vollkommen irrelevant sein sollten.
So hatten wir beispielsweise bereits acht Effekte ''erfunden'', bevor es einen Prototyp gab.\\
Ohne es zu wissen, haben wir das Wasserfallmodell angewandt. Das Modell wird in der Informatik teils als eher schlecht erachtet, da man sich in der modernen
Entwicklung eher auf einen agilen Arbeitsprozess fokussieren sollte. Im Nachhinein müssen wir zugeben, dass wir fast allen möglichen Problemen des Wasserfallmodells
über den Weg gestolpert sind (\url{https://de.wikipedia.org/wiki/Wasserfallmodell#Probleme_und_Nachteile}).


\subsection*{Priorisierung}
Wir hatten eine teils ungenaue Priorisierung. Obwohl wir bereits früh unsere Anforderungen festlegten, haben wir uns in das eine oder andere Detail verbissen.
Dies führte dazu, dass die Entwicklung von wichtigen Anforderungen hinter unwichtigeren Funktionen anstand. 


\subsection*{Manpower} \label{subsec:neusm}
Wir hatten gegenüber einer standardmässigen Maturitätsarbeit den Vorteil, dass wir zu zweit mehr Manpower mitbringen. Doch selbst unserem dynamische Duo sind zu zweit Grenzen gesetzt.
Aus diesem Grund sind wir uns einig, dass eine dritte Person beträchtlich bei der Arbeit geholfen hätte. Eine Möglichkeit wäre gewesen, dass das dritte Teammitglied die ganze
gestalterische Arbeit übernommen hätte. Doch auch ein*e weitere*r Entwickler*in hätte massgebend bei der Arbeit ausgeholfen. Zudem denken wir, dass ein Team mit drei Mitgliedern
immer noch recht klein ist. Aus diesem Grund hätte ein*e dritte*r Arbeitspartner*in zu keinen grossen organisatorischen Problemen geführt.


\subsection*{Dokumentation}
Wir haben den Beginn der schriftlichen Arbeit zu spät angesetzt. Wir hatten den Zeitaufwand für die Strukturierung und gemeinsame intensive Überarbeitung massiv unterschätzt.
Entsprechend kamen wir gegen Ende ins Schwitzen. Ein bekanntes Phänomen in der Spielentwicklung, das Crunchtime (\url{https://en.wikipedia.org/wiki/Crunch_(video_games)}) genannt wird.

\section{Nächstes Mal}

\subsection*{Treffen}
Eine grössere Anzahl an Treffen und Telefonaten würde das gemeinsame Verständnis zum aktuellen Stand verbessern. Gäbe es eine weitere Arbeit, würden wir uns mindestens monatlich treffen, um den Fortschritt und die nächsten Arbeiten zu besprechen.\\
Mögliche Fragen, die auf diesem Weg einfacher zu klären sind, sind:
\begin{itemize}
    \item[-] ''An welchem Punkt stehen wir genau?''
    \item[-] ''Was haben die einzelnen Teammitglieder in der nächsten Zeit vor?''
    \item[-] ''Gibt es neue wichtige Ideen für das Spiel?''
    \item[-] ''Sind unsere alten Skizzen noch wahrheitsgetreu?''
    \item[-] ''Benötigen wir neue Skizzen?''
    \item[-] ''Was sind die nächsten Schritte?''
\end{itemize}


\subsection*{Vier-Augen-Prinzip}
Üblich bei der Nutzung von GitHub als Team sind sogenannte Pull-Requests. Diese finden statt, wenn ein Mitglied eine Änderung auf den Hauptordner übertragen will. Ein solcher
Pull-Request muss dann von einem weiteren Teammitglied genehmigt werden. Der Grundgedanke ist, dass eine zweite Person den Code überprüft und damit mögliche Fehler sowie Verbesserungen erkennt.
Damit werden Fehler reduziert und somit auch Zeit gespart. In einer weiteren Arbeit würden wir dieses Prinzip ohne Ausnahmen anwenden.


\subsection*{Priorisierung}
Diesen Punkt anzupassen und anzuwenden ist eher schwierig. Mit einer grösseren Menge an Treffen, Besprechungen und Planung könnte man dieses Problem allerdings in der Zukunft
minimieren.


\subsection*{Verschwendete Zeit}
Ein Zeitverlust durch Fehler sehen wir nicht als ''verschwendet'', sondern eher als ''Zeit des Lernens''. Durch das Beheben unserer Fehler haben wir sehr wahrscheinlich am meisten gelernt.\\
Wir haben allerdings den Fehler gemacht, dass wir uns zu Beginn acht verschiedene Effekte erarbeitet haben. Dies hat uns wertvolle Planungszeit gekostet, die wir an anderen Stellen besser hätten nutzen können.

