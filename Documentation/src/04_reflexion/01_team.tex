\chapter{Team}

\section{Rückblick}

\subsection*{Teamarbeit}
Wir verstanden uns während der ganzen Arbeit sehr gut. Es kam nie zu zwischenmenschlichen Problemen. Jedoch haben wir viel Potenzial der Teamarbeit nicht ausgeschöpft.
Teilweise war der Arbeitsprozess leider vergleichbar zu dem zweier Schüler, die am gleichen Projekt arbeiten. Dies hat zur Folge, dass es teilweise Code gibt,
den nur einer von uns richtig gut versteht. Diese ''Spezialitäten'' führten teils zu einem recht getrennten Arbeitsprozess. Zurückführen könnte man dies auch auf die zu seltenen
Treffen. Hätten wir uns öfter als dreimal getroffen, wären viele Unklarheiten oder Missverständnisse nicht aufgetreten. So waren wir beide des Öfteren unsicher, wie genau wir unsere verrichtete Arbeit
dokumentieren sollten. Es stellte sich zudem die Frage, was genau als investierte Zeit geltet.
Unser grösster Fehler war allerdings, dass wir keinen Schutzmechanismus für die Fehler einer Person eingeführt hatten. So war es für uns beide möglich, direkt am Hauptcode Änderungen vorzunehmen.
Hätten wir etwas dagegen vorgenommen, hätten wir keine 5 Stunden wegen eines kleinen Rechtschreibfehlers verloren.


\subsection*{Agilität}
Ein weiterer negativer Punkt unserer Projektplanung ist die entstandene Steifheit. Ein Ziel vor den Augen zu haben und eine Vision zu verfolgen ist wichtig. Jedoch soll es
immer möglich sein, diese anzupassen. Bereits im März haben wir sehr viel getüftelt und Ideen entwickelt, die zu diesem Zeitpunkt vollkommen irrelevant sein sollten.
So hatten wir beispielsweise bereits acht Effekte ''erfunden'', bevor es einen Prototyp gab.\\
Ohne es zu wissen, haben wir das Wasserfallmodell angewandt. Das Modell wird in der Informatik teils als eher schlecht erachtet, da man sich in der modernen
Entwicklung eher auf einen agilen Arbeitsprozess fokussieren sollte. Im Nachhinein müssen wir zugeben, dass wir fast allen möglichen Problemen des Wasserfallmodells
über den Weg gestolpert sind (\url{https://de.wikipedia.org/wiki/Wasserfallmodell#Probleme_und_Nachteile}).


\subsection*{Priorisierung}
Wir hatten im Allgemeinen eine teils ungenaue Priorisierung. Obwohl wir bereits ziemlich früh unsere Anforderungen festlegten, haben wir uns zum Teil in das eine oder andere verbissen.
Dies führte dazu, dass unter Umständen die Entwicklung von wichtigen Anforderungen für unwichtigere Funktionen pausiert wurde. 


\subsection*{Manpower} \label{subsec:neusm}
Wir hatten gegenüber einer standardmässigen Maturitätsarbeit den Vorteil, dass wir zu zweit mehr Manpower mitbringen. Doch selbst unserem Duo sind zu zweit Grenzen gesetzt.
Aus diesem Grund sind wir uns einig, dass eine dritte Person beträchtlich bei der Arbeit geholfen hätte. Eine Möglichkeit wäre gewesen, dass das dritte Teammitglied die ganze
gestalterische Arbeit übernommen hätte. Doch auch ein weiterer Entwickler hätte massgebend bei der Arbeit ausgeholfen. Zudem denken wir, dass ein Team mit drei Mitgliedern
immer noch recht klein ist. Aus diesem Grund hätte ein dritter Arbeitspartner zu keinen grossen Team-Management-Problemen geführt.


\subsection*{Dokumentation}
Wir haben den Beginn der schriftlichen Arbeit zu spät angesetzt. Zu Beginn hatten wir eine sehr genaue, fast schon abgabebereite Dokumentation. Mit der Zeit ging dies allerdings verloren.
Gegen Ende stand nun in der Zeitnotierung lediglich die Funktion, die nicht funktioniert hatte, und nicht wie zu Beginn eine genauere Dokumentation die darlegt; was wo wieso genau nicht funktioniert hat.

\section{Nächstes Mal}

\subsection*{Treffen}
Eine höhere Anzahl an Treffen und Telefonaten würden eine höhere Klarheit bedeuten. Gäbe es eine weitere Arbeit, würden wir uns mindestens monatlich treffen, um den Stand der Dinge zu besprechen.\\
Mögliche Fragen, die auf diesem Weg einfacher zu klären sind, sind:
\begin{itemize}
    \item[-] ''An welchem Punkt stehen wir genau?''
    \item[-] ''Was haben die einzelnen Teammitglieder in der nächsten Zeit vor?''
    \item[-] ''Gibt es neue wichtige Ideen für das Spiel?''
    \item[-] ''Sind unsere alten Skizzen noch wahrheitsgetreu?''
    \item[-] ''Benötigen wir neue Skizzen?''
    \item[-] ''Was sind die nächsten Schritte?''
\end{itemize}


\subsection*{Vier-Augen-Prinzip}
Üblich bei der Nutzung von GitHub als Team sind sogenannte Pull-Requests. Diese finden statt, wenn ein Mitglied eine Änderung auf den Hauptordner übertragen will. Ein solcher
Pull-Request muss dann von einem weiteren Teammitglied genehmigt werden. Der Hintergrundgedanke ist, dass das zweite Mitglied den Code ebenfalls auf mögliche Fehler untersucht.
Dieser Weg reduziert die Anzahl an Fehlern und erhöht somit auch die schlussendlich gesparte Zeit. In einer weiteren Arbeit würden wir dieses Prinzip ohne Ausnahmen anwenden.


\subsection*{Priorisierung}
Diesen Punkt anzupassen und anzuwenden ist eher schwierig. Mit einer grösseren Menge an Treffen, Besprechungen und Planung könnte man dieses Problem allerdings in der Zukunft
minimieren.


\subsection*{Verschwendete Zeit}
Die gesamte ''verschwendete Zeit'' beträgt lediglich wenige Stunden. Trotzdem haben wir zu Beginn den Fehler getätigt und zu weit in die Zukunft geplant. Ein Zeitverlust durch Fehler
sehen wir nicht als ''verschwendet'', sondern eher als ''Zeit des Lernens''. Durch das Beheben unserer Fehler haben wir sehr wahrscheinlich am meisten gelernt.
Es würde allerdings mehr Sinn ergeben, sich zu Beginn zu entscheiden, dass Effekte existieren werden. Zudem könnte man sich eventuell bereits ein Proof of Concept ausdenken.
Kommt man in der Entwicklung in ein Stadium, in dem die Zeit es erlaubt, kann man immer noch mehrere Ideen für Effekte erarbeiten. \\
Wir haben allerdings den Fehler gemacht, dass wir uns zu Beginn acht verschiedene Effekte erarbeitet haben. Dies hat uns wertvolle Planungszeit gekostet, die wir an anderen Stellen besser hätten nutzen können.