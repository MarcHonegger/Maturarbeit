\chapter{Team}

\section{Rückblick}

\subsection*{Kommunikation}
Wir verstanden uns während der ganzen Arbeit sehr gut.
Es kam nie zu zwischenmenschlichen Problemen. Jedoch haben wir viel Potenzial der Teamarbeit nicht ausgeschöpft.
Eigentlich war es mehr zwei Schüler die am gleichen Projekt arbeiten.
Natürlich haben wir die Idee zusammen erarbeitet, haben uns getroffen und hatten das selbe Ziel, jedoch wars das.
Wir hatten kein Schutzmechanismus für Fehler einer Person.
So haben wir einmal etwa 5h verloren wegen einem kleinem Schreibfehler.
Wir haben uns relativ wenig getroffen, einmal für die Idee, einmal für das Timemanagemnt und zweimal für sonstiges.
Wir haben viel Code im Spiel, welcher nur einer von uns versteht.
Wir waren beide häufig unsicher wie wir Dinge dokumentieren und was alles als investierte Zeit giltet.

\subsection*{Agilität}
Ein anderer Punkt ist die Steifheit unseres Projekts.
Ein Ziel und eine Vision ist wichtig, jedoch sollte sie jederzeit angepasst werden können.
Wir haben zu Begin sehr viel getüftelt, das auch schon an Dingen die zu Beginn völlig irelevant sind.
So haben wir z.B. 8 Effekte erfunden, bevor wir überhaupt ein Spiel hatten.
In der Informatik nennt sich das "das Wasserfall-Prinzip" und wird als schlecht erachtet.
In der modernen Entwicklung fokusiert man sich auf das agile entwicklen.

\subsection*{Priorisierung}
Wir hatten eine schlechte Priorisierung.
So hat unser Spiel z.B. ein schönes UI und ein funktionsfähiges Pausenmenu, aber es fehlen einige Ingame Features, wie z.B. das Deck bauen.

\subsection*{Manpower}
Uns sind einfach Grenzen gesetzt zuzweit.
Wir sind uns einig, eine dritte Person hätte geholfen.
Entweder eine gestalterische Person, welche die ganze Gestaltung übenrommen hätte, oder ein weiter Entwickler.
Wir denken auch, drei Leute ist immernoch ein kleines Team und es sollte noch zu keinen grossen Team management Problemen führen.

\subsection*{Dokumentation}
Wir haben die Dokumentation zu weit nach hinten geschoben.
Anfangs hatten wir eine sehr genaue und teils Maturaarbeit reife Dokumentation, jedoch ging dies später verloren.
So sind viele Zeitaufwandeinträge ohne Kommentar oder mit sehr wenig Erläuterung.

\section{Nächstes Mal}

\subsection*{Treffen}
Mehr Treffen, Calls und Klarheit.
Bei einer weiteren Arbeit würdne wir uns ein mal monatlich treffen um einfach über den Stand der Dinge zu reden.
Wo genau sind wir jetzt?
Wer hat was in nächster Zeit vor?
Gibt es Features die wir vergessen haben?
Gibt es neu Ideen für Features?
Sind usnere Skizzen noch Zeit getreu?
Braucht es Neue?
So soll das Spiel Schritt für Schritt wachsen

\subsection*{Vier-Augen-Prinzip}
Üblich in der Entwicklung mit GitHub sind sogenannte Pullrequests.
Falls ein Mitglied änderen vornimmt, macht er eine Anfrage für seine Anpassungen.
Diese müssen dann von jemand anderem genehmigt werden.
So sollen Fehler reduziert und schlussendlich Zeit gesparrt werden.
Dieses Prinzip würden wir in einer weiteren Arbeit auch anwenden.

\subsection*{Priorisierung}
Dieser Punkt ist schwierig anzupassen.
Mit mehr Treffen, Besprechung und Planun sollte dieses Problem sicher minimiert sein.

\subsection*{Verschwendete Zeit}
Zwar handelt sich nur um wenige Stunden, dennoch haben wir viel in die Zukunft geschaut.
So macht es zum Beispiel Sinn sich zu entscheiden, dass Effekte exisiteren und sich als Proof-of-Concept einen auszudenken, jedoch nicht acht.
Diese können laufend, einmal so weit gekommen, durchdacht und entwickelt werden.