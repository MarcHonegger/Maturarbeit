\chapter{Marc}

\section{Rückblick}
Ich bin äusserst zufrieden mit unserer Arbeit.
Allerdings war meine ursprüngliche Vorstellung unseres Spiels ausgereifter.
Dies scheiterte aber nicht an der Arbeit von Elia, mir oder uns beiden, sondern einfach an meinem Hochmut.
Ich habe vor Jahren versucht, die selbe Spielidee auf eigene Faust umzusetzen und kam in relativ kurzer Zeit sehr weit.
Ich dachte der Fortschritt wächst linear.
Jedoch stiessen auch wir hier auf das Paretoprinzip: 20\% Aufwand gleich 80\% Ergebnis und die restlichen 20\% Ergebnis benötigen 80\% des Aufwands.
Der Feinschliff des UI, der Abschluss der schriftlichen Arbeit und ein bugfreies Spiel waren riesige Zeitfresser. \\
Die Arbeit mit Elia war voll und ganz erfreulich.
Es gab keine Probleme und es hat mir durchgehend Spass gemacht.
Wir hatten zwar wenig konkrete Gespräche, dennoch fanden wir uns immer wieder im Gespräch über unsere Arbeit.
Ich bin extrem froh Elia als Partner angefragt zu haben, und ich denke er war für das Spiel mindestens gleich bereichernd wie ich. \\
Gegen Ende wurde die Arbeit doch noch stressig.
Ich kann nur von Glück reden, dass Elia auch im Endspurt so ein grosses Engagement.
Für mich war es bis kurz vor Schluss sehr schwierig wieder in den Flow zu kommen.

\section{Nächstes Mal}
Vieles würde ich genauso wiederholen:
zum Beispiel den Start unserer Arbeit, die Arbeit an sich und die Wahl meines Partners.
Aber natürlich lief vieles nicht reibungslos.
Das grösste Problem war mangelnde Planung und Routine.
Folgende 'Systeme' sollen dem bei meinem nächsten Projekt entgegenwirken:
\begin{enumerate}
    \item Ich werde mir ein Stundenminimum pro Woche setzen.
    Dies soll nicht die angepeilten Anzahl Stunden darstellen, sondern kontinuirliches Arbeiten garantieren.
    So wird verhindert, dass ich komplett aus der Arbeit rausfalle und Schwierigkeiten habe wieder reinzufinden.
    \item Ich werde ein Reflexionsystem einführen.
    In diesem soll wöchentlich oder allenfalls monatlich reflektiert werden.
    Was lief gut? Was lief schlecht? Was kann ich besser machen?
    \item Ich werde wöchentlich brainstormen, alleine oder zu zweit. 
    aber einmal weg vom Codezeilen schreiben.
    Dies würde dem Projekt deutlich mehr Agilität und Anpassung bringen.
    Ich bin auch sicher, dies führt zu vielen und sicher auch interessanten Ideen.
\end{enumerate}
Diese Punkte hätten alle dabei geholfen, mir das Ziel vor Augen zu führen.
Denn genau dieses Ziel verlor ich zeitweise aus dem Blick.
Schlussendlich hatte ich weniger Spass und bekam Schwierigkeiten mich zu motivieren. \\
Da unser Projekt nicht mit der Abgabe abgeschlossen ist, werde ich gemäss dieser Reflektion die oben genannten Stategien versuchen umzusetzen. 
Nichtsdestotrotz würde ich beim nächstem mal eine dritte Person mit ins Boot holen.\\
Trotz der teilweise noch fehlenden Features bin ich stolz auf uns und denke wir haben für zwei Gymischüler ein brilliantes Produkt erstellt.