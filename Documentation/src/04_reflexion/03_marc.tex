\chapter{Marc}

\section{Rückblick}
Ich bin äusserst zufrieden mit unserer Arbeit.
Allerdings war meine ursprüngliche Vorstellung unseres Spiels ausgereifter.
Dies scheiterte aber nicht an der Arbeit von Elia, mir oder uns beiden, sondern einfach an meinem Hochmut.
Ich habe vor Jahren die selbe Spielidee auf eigene Faust versucht umzusetzen und kam in relativ wenig Zeit sehr weit.
Ich dachte der Fortschritt wächst Linear.
Jedoch stiessen auch wir hier auf das Paretoprinzip: 20\% Aufwand gleich 80\% Ergebnis und die restlichen 20\% Ergebnis benötigen 80\% des Aufwands.
Der Feinschliff des UI-Aussehen, der Abschluss der schriftlichen Arbeit und ein bugfreies Spiel waren relativ zu ihrer Grösse riesen Zeitfresser. \\
Die Arbeit mit Elia war voll und ganz erfreulich.
Es gab keine Probleme und es hat mir durchgehend Spass gemacht.
Wir hatten zwar wenig konkrette Gespräche, dennoch fand man sich immer wieder im Gespräch über unsere Arbeit.
Ich bin extrem froh Elia als Partner angefragt zu haben und ich denke er war für das Spiel mindestens gleich bereichernd wie ich. \\
Gegen Ende wurde die Arbeit doch noch zu einem Stress.
Ich kann nur von Glück reden, dass Elia im Endspurt so eine riesen Arbeit geleistet hat.
Für mich war es bis kurz vor Schluss sehr schwierig wieder in den Flow zu kommen.

\section{Nächstes Mal}
Vieles würde ich genauso wiederholen.
So zum Beispiel der Start unserer Arbeit, die Arbeit an sich und die Wahl meines Partners.
Aber natürlich lief fieles nicht reibungslos.
Das grösste Problem war mangelnde Planung und Routine.
Folgende 'Systeme' sollen dem bei meinem nächsten Projekt entgegenwirken:
\begin{enumerate}
    \item Ich werde mir ein Stundenminimum pro Woche setzen.
    Dies soll nicht die angepeilten Anzahl Stunden darstellen, sondern nur ein Start und somit maximal wenige Stunde pro Woche betragen.
    Dies soll verhindern, dass ich komplett aus der Arbeit rausfalle und Schwierigkeiten habe wieder reinzukommen.
    \item Ich werde ein Reflexionsystem einführen.
    In diesem soll wöchentlich oder allenfalls monatlich reflektiert werden.
    Was lief gut? Was lief schlecht? Was kann ich besser machen?
    \item Ich werde wöchentlich Brainstormen.
    Soll es allein oder zu zweit sein, aber einmal weg vom Codezeilen schreiben.
    Dies würde dem Projekt deutlich mehr Agilität und Anpassung bringen.
    Ich bin auch sicher, dies führt zu vielen und sicher auch interessanten Ideen.
\end{enumerate}
Diese Punkte hätten mir alle dabei geholfen mir das Ziel vor Augen zuführen.
Denn genau dieses Ziel verlor ich zeitweise.
Schlussendlich hatte ich weniger Spass und bekam Schwierigkeiten mich zu motivieren. \\
Da unser Projekt nicht mit der Abgabe abgeschlossen ist, werde ich nach dieser Reflektion meinen Zielen und Plänen widmen. \\
Entgegen der unerfüllten Erwartung bin ich stolz auf uns und denke wir haben für zwei Gymischüler ein brilliantes Produkt erstellt.
Nichtdestotrotz würde ich beim nächstem mal eine dritte Person mit ins Boot holen.