\chapter{Elia}

\section{Rückblick}
Das Produkt dieser Arbeit erfreut mich sehr. Obwohl dem Prototyp unseres Videospiels noch einige Funktionen fehlen, bin ich doch grundsätzlich mit unserem Spiel zufrieden. \\
Dies ist sehr wahrscheinlich das grösste und zeitaufwendigste Projekt, an dem ich jemals gearbeitet habe. Unser GitHub Repository enthält mehr als 40'000 Zeilen an C\# Source-Code.
Das ganze Repository ist ca. 1.1 Millionen Zeilen lang und mehrere Gigabytes gross (einschliesslich DLL's etc.). \\
Alleine hätte ich dies nie geschafft, deshalb bin ich besonders froh, dass wir dies im Team verwirklichen konnten. Die Gruppenarbeit hat mir viel Druck von den Schultern genommen, da ich mich bei Problemen immer auf Marc
verlassen konnte. Zudem führte die Zusammenarbeit zu vielen guten Ideen, die wir auch nur zusammen als Team in die Tat umsetzen konnten. Durch diese Gruppenarbeit wurde unsere Freundschaft stark gestärkt.
Ich habe zu schätzen gelernt, dass ich mich die ganze Zeit voll und ganz auf Marc verlassen konnte.\\
Obwohl von Anfang an klar war, dass dies eine sehr aufwendige Arbeit sein würde, habe ich sie doch unterschätzt. Am meisten unterschätzt habe ich allerdings den Arbeitsaufwand der schriftlichen Arbeit. Diese stellte sich als ziemlich grosser Zeitschlucker heraus. \\
Ich hatte zudem unterschätzt, was mein geplantes Hardwareupgrade für immensen Einfluss auf meinen Arbeitsfluss haben würde. Der Wechsel von meinem Laptop auf einen hochmodernen Computer brachte viele Vorteile mit sich.
Die vervielfachte Leistung hatte den Vorteil, dass ich bei Kompilierung des Codes keine halbe Minute mehr warten musste. Stattdessen konnte ich zwei Sekunden später unser Spiel testen.
Diese immense Rechenleistung merkte man auch daran, dass sich mein Zimmer durch die Nutzung des Computers im Laufe eines Tages trotz geöffnetem Fensters um mehrere Grade erwärmte.\\
Auch wenn die Fehlersuche teilweise sehr anstrengend war, hat mir das Programmieren trotz teilweise langen Tagen und Nächten meistens Spass gemacht. Ich fand es trotzdem schade,
dass ich teilweise den ganzen Tag nur im Zimmer sass und an der Maturaarbeit programmierte. 

\section{Nächstes Mal}
Für das nächste Mal möchte ich mir zu Beginn des Semesters einen Plan erstellen, der festlegt, dass ich pro Woche mindestens eine Stunde arbeiten werde. Mit diesem Plan wäre es auf keinen Fall zu dieser sehr langen
Pause (von Mai bis August) gekommen. Wenn ich erneut die Chance hätte, dieses Projekt in einer Gruppenarbeit zu lösen, würde ich diese Chance ohne zu zögern ergreifen. Allerdings würde ich in Zukunft
unsere Organisation zwingend verbessern. Eventuell wäre ein drittes Teammitglied auch keine schlechte Idee (vgl. \autoref{chap:neusm}).\\
Was diese Erfahrung allerdings mit meinem Plan, Informatiker zu werden, geändert hat, kann ich zu diesem Zeitpunkt nicht beantworten. Fakt ist, ich habe nun auf jeden Fall eine genauere Vorstellung,
wie das Leben eines Informatikers aufgebaut ist. Zudem habe ich eine Menge an Respekt für jeden Spielentwickler gewonnen. Ein Spiel zu entwickeln ist nicht leicht, deswegen werde ich ab nun immer meine Erfahrungen im Hinterkopf
behalten und auch immer an das Dev Team denken, wenn ich über einen Bug oder Crash fluche.