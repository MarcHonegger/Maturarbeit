% Maturarbeit game documentation

\documentclass[11pt, oneside, a4paper]{book}

\usepackage[german]{babel}
\usepackage{quoting,xparse}


\NewDocumentCommand{\bywhom}{m}{% the Bourbaki trick
  {\nobreak\hfill\penalty50\hskip1em\null\nobreak
   \hfill\mbox{\normalfont(#1)}%
   \parfillskip=0pt \finalhyphendemerits=0 \par}%
}

\NewDocumentEnvironment{pquotation}{m}
  {\begin{quoting}[
     indentfirst=true,
     leftmargin=\parindent,
     rightmargin=\parindent]\itshape}
  {\bywhom{#1}\end{quoting}}

\input{custom}

\makeglossaries


\newglossaryentry{Feature}
{
  name={Feature},
  description={Eine bestimmte Funktion, welche die Software weiterbringt}
}
\newglossaryentry{Online}
{
  name={Online},
  description={Nicht nur lokal auf dem eigenen Rechner, sondern im Internet}
}

\newglossaryentry{Lokal}
{
  name={Lokal},
  description={Im selben Netz oder Computer}
}

\newglossaryentry{FPS}
{
  name={FPS},
  description={Frames Per Second. Anzahl Bilder pro Sekunde. Höhere FPS ist meistens gleichbedeutend zu einem flüssigeren Spielerlebnis}
}

\newglossaryentry{Recode}
{
  name={Recode},
  description={Komplette Umprogrammierung und Reprogrammierung einer Funktion}
}

\newglossaryentry{Crossplay}
{
  name={Crossplay},
  description={Plattformübergreifendes Spielen. Die Möglichkeit das gleiche Spiel auf zwei unterschiedlichen Plattformen zu spielen (hier: Windows und macOS)}
}

\newglossaryentry{Anti-Cheat}
{
  name={Anti-Cheat},
  description={Software um Schummeln zu verhindern.
  Ist sehr komplex und bei allen grösseren online Videospielen vorhanden.
  Teilweise sehr tief in das System vernetzt}
}

\newglossaryentry{Skins}
{
  name={Skins},
  description={Kaufbare Darstellungen, welche von dem Standard abweichen.
  In vielen Spielen nur im Tausch mit echter Währung zu erhalten und bringen nur ästhetische Unterschiede}
}

\newglossaryentry{Ingame}
{
  name={Ingame},
  description={Im laufendem Spiel, nicht auf dem Computer oder beim Start oder Schliessen des Spiels}
}

\newglossaryentry{Pay-to-Win}
{
  name={Pay-to-Win},
  description={Übersetzt: Bezahlen-zum-Gewinnen, bedeutet, dass mit echter Währung Ingame Vorteile erkauft werden können}
}

\newglossaryentry{Firewall}
{
  name={Firewall},
  description={Schutzsystem eines technischen Systems um Viren und ähnliches fernzuhalten.
  Blockiert aus Sicherheit manchmal mehr als nötig und erschwert so zum Beispiel das Verbinden von zwei Computern über das Internet}
}

\newglossaryentry{Shop}
{
  name={Shop},
  description={ein Ort im Spiel, in dem Dinge gekauft werden können. Ein Ingame-Laden}
}


\newglossaryentry{Multiplayer}
{
  name={Multiplayer},
  description={Mehrspieler, also mehrere Spieler spielen zusammen das gleiche Spiel}
}

\newglossaryentry{Singleplayer}
{
  name={Singleplayer},
  description={Einzelspieler, das Spiel wird alleine, meist nur lokal, gepielt}
}

\newglossaryentry{Tutorial}
{
  name={Tutorial},
  description={Eine Anleitung, in Videospielen meist eine kurze Spielsequenz, welche Tipps und eine fixe Abfolge besitzt}
}

\newglossaryentry{Helden}
{
  name={Helden},
  description={In Videospielen der wichtigste Charakter.
  In gewissen Spielen ist der Held der Charakter, den man spielt.
  Der Tod des Helden bedeutet einen massiven Nachteil}
}

\newglossaryentry{Kamera}
{
  name={Kamera},
  description={Die Kamera nimmt in Unity auf, was der Spieler sieht.
  Mit Kamera ist also das gemeint, was der Spieler sieht}
}

\newglossaryentry{Lane}
{
  name={Lane},
  description={
    Bezeichnet generell eine Linie oder Bahn auf der gegangen werden kann.
    Sehr prägnant in MOBAs wie League of Legends oder Dota.
    In unserem Spiel eine Bahn der vier, auf welcher Truppen beschwört werden können}
}

\newglossaryentry{Steam}
{
  name={Steam},
  description={
    Steam ist der grösste Videospiel-Anbieter der Welt.
    Gehört dem Unternehmen Valve}
}

\newglossaryentry{Indie Developer}
{
  name={Indie Developer},
  description={
    Ein Indie Developer entwickelt Indie Games, kurz für "Independent Game".
    Diese werden von einer einzelnen Person oder kleinen Game Studios entwickelt.
    Der Gegensatz sind \gls{AAA-Spiel}e}
}

\newglossaryentry{AAA-Spiel}
{
  name={AAA-Spiel},
  description={
    Ausgesprochen "Tripple A", sind  Videospiele mit maximalem Budget und meist erstellt von den grössten \glspl{Game Studio}}
}

\newglossaryentry{Game Studio}
{
  name={Game Studio},
  description={
    Ein Unternehmen, das Videospiele entwickelt}
}

\newglossaryentry{Host}
{
  name={Host},
  description={Die Person die das Spiel hostet. Sie ist Server und Client}
}

\newglossaryentry{Client}
{
  name={Client},
  description={Die Person, die dem Server einer anderen Person als Mitspieler beitritt}
}

\newglossaryentry{GitHub Repository}{
  name={GitHub Repository},
  description={Ein Repository enthält alle Ordner und Dateien des aktuelles Projekts. Innerhalb dieses Repository kann man die ganze bisherige Arbeit anschauen und kontrollieren. Dabei speichert das Repository
  auch eine Vergangenheit jeder Datei, sodass man Änderungen einfach zurücksetzen kann}
}

\newglossaryentry{DLC}{
  name={DLC},
  description={DLC steht für ''Downloadable Content'', also zusätzlich herunterladbare Inhalte. Diese bieten eine Erweiterung zu der ''Standard-Version''. DLCs sind teilweise recht teuer}
}
\newglossaryentry{Cheater}
{
  name={Cheater},
  description={Jemand, der unerlaubte Methoden oder Programme verwendet, um sich einen unfairen Vorteil zu verschaffen.
  Auf Deutsch \textit{ein Schummler}}
}
\newglossaryentry{Paretoprinzip}
{
  name={Paretoprinzip},
  description={20 Prozent der Zeit für 80 Prozent der Arbeit, 80 Prozent der Zeit für 20 Prozent der Arbeit (\url{https://de.wikipedia.org/wiki/Paretoprinzip})}
}



% \newglossaryentry{slalom}{
%     name={Slalom},
%     description={A kind of \gls{contract} in the \gls{coiffeur-jass} which plays without a trump.
%     Alternates between \gls{bottoms-up} and \gls{tops-down} or vice-versa, scoring rules apply from the first chosen option}
% }


\hypersetup{
    pdfauthor={Marc Honegger, Elia Schürpf},
    pdftitle={Maturarbeit Game Documentation},
}

\begin{document}

\pagestyle{empty}

\frontmatter

\begin{titlepage}

    \begin{center}

        \vspace{1 cm}

        {\Large Maturitätsarbeit 2023 \\ Dokumentation} \\

        \vspace{1 cm}

        \vspace{0.5cm}

        {\Huge Laneclash}

        \vspace{0.5cm}

        \vspace{1 cm}

        \includegraphics[height=7cm]{resources/laneclash.png}

        \vspace{1 cm}

        Version: 0.1 \\
        Datum: \DTMnow \\
        \vspace{1 cm}

        \begin{tabular}{rl}
            \textbf{Team:}          & Marc Honegger (C6a) \\
                                    & Elia Schürpf (C6a)\\
                                    \\
            \textbf{Begleitperson:} & Martin Hunziker
        \end{tabular}

        \vfill

        \vspace{1cm}
        Gymnasium \\
        KZO - Kantonsschule Zürcher Oberland

    \end{center}

\end{titlepage}

\tableofcontents

\mainmatter

\part{Ziel}
\chapter{Vision}
\begin{center}
Ein funktionsfähiges Videospiel, welches jede vergleichbare Maturitätsarbeit in den Schatten stellt\\
\end{center}
Die Grundidee für dieses Game war nicht neu, sondern hatte sich vor einigen Jahren in Marcs Kopf geformt.
Jedoch war für die Umsetzung ein Team notwendig. Zusammen mit Elia bildete sich ein gamebegeistertes, programmieraffines und engagiertes Duo.\\
Uns war dennoch von Anfang an bewusst,
dass wir auch zu zweit nicht alle Funktionen erfolgreich umsetzen werden. Dennoch haben wir uns dieses hohe Ziel gesetzt. Interessiert und 
unterstützend begleitete uns Herr Martin Hunziker(IT-Koordinator der KZO).
Vieles konnten wir schlussendlich auch umsetzen. \gls{aaa-spiel}
Das grösste Wagnis war der Multiplayer und das haben wir dann auch an eigenem Leib erfahren.
Alleine für die Erstellung des Multiplayer-Modus investierten wir über 100 Stunden.\\
Letztendlich haben wir es geschafft, ein funktionsfähiges und lustiges Spiel zu entwickeln.
Einige Kommentare:
\begin{center}
    'Sick Game' - Marc Honegger \\
    'Absolut Krass' - Elia Schürpf \\ 
    'Masterpiece' - Martin Hunziker
\end{center}
Wie das Spiel jetzt aussieht, wie es zur aktuellen Version kommen konnte, aber auch wie es in Zukunft damit weiter geht, ist in dieser Arbeit beschrieben.
\chapter{Anforderungen}

\section{KZO}
% \includegraphics[height=18cm]{resources/diagrams/use-case}
Die Anforderungen der Schule halten sich in Grenzen. Zum einen gibt es nicht fast keine
genauen Anforderungen und die restlichen sind sehr tief gehalten. Die drei Anforderungen der KZO sind:
\begin{enumerate}
    \item \textbf{Zeit} \\
        Geplant ist ungefähr eine Lektion pro Woche, denn diese wir dem Stundenplan für das Semester abgezogen.
        Jedoch ist diese Anforderungen bei uns nicht relevant, denn unser Projekt ist eine riesige Maturaarbeit.
        Wir werden ohne Probleme diese Zeitanforderung erreichen und auch bei Weitem überschreiten.
    \item \textbf{Begleitperson} \\
        Jede Maturaarbeit braucht eine Begleitperson. Also eine Lehrperson, welche die Arbeit bewertet und einen unterstützt.
        Letzteres, also Unterstützung, war bei uns sehr schwierig, denn keine Lehrperson kennt sich mit Unity und C\# genug gut aus.
        Wir sind schon früh auf Herr Kern, ein Informatiklehrer, zugegangen.
        Er hat uns an Martin Hunziker, Leiter der IT, weitergeleitet. Er nahm uns sehr gern auf, mit der Anmerkung, dass unsere Arbeit sehr schwierig sein wird, umzusetzen.
    \item \textbf{Plagiat} \\
        Arbeiten dürfen nicht einfach kopiert werden. Wir haben uns zwar bei gewissen Problemen online inspirieren lassen, jedoch sind diese keine Plagiate.
        Der grösste Teil unseres Codes und alles unserer schriftlichen Arbeit sind von uns geschrieben.
\end{enumerate}

\section {Unsere}
Gewisse Features sind wichtiger als andere für das Funktionieren des Spiels.
Zum Beispiel sind Einstellungen, Truppen und Multiplayer notwendiger als die Monetarisierung, zumindest nach uns.
Deshalb ist dieser Abschnitt in 4 Gruppen unterteilt:
\begin{enumerate}
    \item \textbf{Notwendig:}
        Ohne diese Features läuft das Spiel nicht oder ist sehr mangelhaft,
        Fehler erfüllt und unspielbar. Ohne ist Idee des Spieles nicht erkennbar.
    \item \textbf{Anvisiert:}
        Diese Features sind unser Ziel. Sie wurden Herr Hunziker, also der Begleitperson, angekündigt und sind alle in einem Plan festgehalten.
        Diese Ziele sollen dazu führen, dass das Spiel
        gut spielbar ist und es Spass macht. Sie sind auch essenziell, das Spiel nicht zu monoton zu gestalten und sollten
        alle bis zum Abgabetermin der schriftlichen Arbeit vollendet sein.
    \item \textbf{Möglichkeiten:}
        Möglichkeiten das Spiel noch auf die Spitze zu bringen. Keine dieser Anforderungen ist nötig
        für das spielen, jedoch können sie das Spielerlebnis spannender und ausgereifter machen. Diese
        Ziele sind entweder in den Sternen geschrieben oder zu erreichen bis zur mündlichen Präsentation.
        Vereinzelt können diese, wie die anvisierten Anforderungen, bis zur schriftlichen Abgabe erreicht werden.
    \item \textbf{Verworfen:}
        Weitere Ideen, welche wir hatten, jedoch als unmöglich oder nicht sinnvoll erachten.
\end{enumerate}

\subsection*{Notwendig}
\begin{itemize}
    \item \textbf{System} \\
        Unser Spiel sollte auf den meisten modernen Geräten funktionieren. Es soll auf macOS und
        Windows fliessend gespielt werden können. Also mit mindestens 60FPS.
    \item \textbf{Benutzeroberfläche} \\
        Eine Startseite mit einem ''Spielen''-Knopf, einem ''Einstellungen''-Knopf, einem
        ''Credits''-Knopf und einem ''Verlassen''-Knopf. Sie soll intuitiv sein und einigermassen
        schön aussehen. Die Benutzeroberfläche soll mit Touch bedienbar sein, andere Eingabemöglichkeiten
        werden nicht unterstützt.
    \item \textbf{Lokaler Multiplayer} \\
        Das Spiel soll im lokalen Netz gespielt werden können. So zum Beispiel mit Freunden oder Familie.
        Es sollte mithilfe der IP-Adresse innerhalb des gleichen Netzes zusammengespielt werden können.
    \item \textbf{Einstellungen} \\
        Auflösung, Vollbild und Lautstärke müssen eingestellt werden können.
    \item \textbf{Kamera} \\
        Die Kamera ist beweglich und das ganze Schlachtfeld ist sichtbar.
    \item \textbf{Truppen} \\
        Es braucht Truppen, die für einen kämpfen können. Diese sind notwendig für den Verlauf des Spieles
        und ohne sie hat das Spiel keinen Sinn.
    \item \textbf{Gewinnmöglichkeit} \\
        Eine Chance das Spiel zu gewinnen oder verlieren und es somit zu beenden. Dies kann sehr simpel
        mit einer Linie vollendet werden, welche beim Überschreiten das Spiel beendet.
    \item \textbf{Lanes} \\
        Unser Spiel ist zwar 2D, aber hat dennoch eine Tiefe.
\end{itemize}

\subsection*{Anvisiert}
\begin{itemize}
    \item \textbf{Helden} \\
        Beide Spieler haben eine Art Truppe, welche ihre Siegeslinie beschützt. Sie haben einen Racheeffekt, welcher die
        Lane ausradiert, damit das Spiel nicht sofort vorbei ist.
    \item \textbf{Design} \\
        Das Spiel sollte vom Aussehen her was hergeben. Es sollte nicht wie ein Prototyp, sondern wie ein 
        vollendetes Spiel aussehen. 
    \item \textbf{Deck} \\
        Zu Beginn Spieler könne ihr eigenes Deck aus einer Auswahl von Karten zusammenstellen. Im Spiel werden davon per Zufall Karten gezogen.
    \item \textbf{Bot} \\
        Ein sehr simpler Algorithmus um alleine gegen den Computer zu spielen auf den Schwierigkeitsstufen \textit{'Einfach', 'Mittel', 'Schwierig'}.
        Es sollen rein zufällig Truppen geschickt werden, bei höherer Schwierigkeit mehr.
    \item \textbf{Truppen}
    \begin{itemize}
        \item \textbf{Nahkampf}
            Eine Truppe mit wenig Reichweite.
        \item \textbf{Fernkampf}
            Eine Truppe die auf Reichweite angreift. Sie scheisst Projektive, zum Beispiel ein Bogenschütze,
            der mit Pfeilen schiesst.
        \item \textbf{Suizid}
            Eine Truppe die bei Berührung mit einem Gegner stirbt und einen Effekt auslöst.
    \end{itemize}
    \item \textbf{Effekte}
    \begin{itemize}
        \item \textbf{Gift:}
            Zeitlich limitierter und wiederholender Schadenseffekt. Eine visuelle Markierung soll vorhanden sein
            und das gleiche Gift, also der gleichen Truppe, soll nur einmal auf jemanden sein.
        \item \textbf{Dorn:}
            Truppen, welche Charaktere mit Dorn attackieren, erhalten schaden. Soll nach Auswahl auf Nahkämpfer,
            Fernkämpfer oder beides wirken.
        \item \textbf{Rache:}
            Truppen mit Rache haben einen Effekt nach ihrem Tod. Zum Beispiel eine Truppe beschwören oder Schaden verursachen.
    \end{itemize}
\end{itemize}

\subsection*{Möglichkeiten}
\begin{itemize}
    \item \textbf{Zauber} \\
        Als Alternative sollen nicht alle Karten einfach eine Truppe herbeirufen. Gewisse Karten sollten nur einen Effekt auslösen.
        Beispiele für Zauber:
            "Heile deine Helden um 5 leben"
            "Gib deinen Helden 5 Rüstung"
            "Verursache allen gegnerischen Truppen 5 Schaden"
            "Ziehe 3 Karten"
    \item \textbf{Monetarisierung} \\
    Hierfür sind uns drei Möglichkeiten eingefallen, hier jeweils die Vor- und Nachteile:
    \begin{enumerate}
        \item \textbf{Kaufbare Gegenstände}
        \begin{itemize}
            \item[+] Vielseitige und Nachhaltige Monetarisierung. Neue Features, bedeuten auch neue Geldmöglichkeit. Die Inhalte sollten
                     auch erspielbar sein, somit kann bezahlen zwar zu Vorteilen führen, jedoch mit viel Spielen trotzdem erreichbar sein.
            \item[-] Pay-to-Win
        \end{itemize}
        \item \textbf{Skins}
        \begin{itemize}
            \item[+] Weit verbreitet und sehr beliebt bei Spielern. Gibt keinem Spieler Vorteile
            \item[-] Wenig Nutzen und sehr aufwändig. Neue Designs zu erstellen, wäre bei uns nicht so sinnvoll.
                     Wir haben noch sehr viel Features zu implementieren und sind nicht Designer. Für uns braucht es sehr
                     viel Zeit eine brauchbare Darstellung zu erstellen und der Mehrwert, welcher dem Spiel beigetragen wird, ist marginal.
        \end{itemize}
        \item \textbf{Kostenpflichtiges Spiel}
        \begin{itemize}
            \item[+] Einmalige Bezahlung, was für viele Nutzer lukrativer ist. Jedoch gilt dies vor allem bei Singleplayer Spielen.
            \item[-] Wir nehmen an, dass wenige Leute bereit wären, Geld für unser Spiel zu bezahlen.
                    Dafür wird es nicht genug ausgereift sein.
                    Auch ist diese Methodik nicht nachhaltig und führt nur zu einer einmaligen Geldspritze.
                    Viele grössere Spiele führen deshalb später
                    DLCs ein, um das Spiel zu erweitern.
                    Jedoch ist dies bei einem Multiplayer Spiel Pay-to-Win.
                    Abschliessen müssten wir höchstwahrscheinlich selbst Geld vorauswerfen, um unser Spiel anbieten zu können, z.B. auf Steam.
        \end{itemize}
        
    \end{enumerate}
    \item \textbf{Online Multiplayer} \\
        Besser als Multiplayer limitiert auf dasselbe Netz, ist Multiplayer, der weltweit verfügbar ist. Jedoch ist von einem Computer auf einen anderen zu
        verbinden dank Firewall von Router und Computer, nahezu unmöglich. Dies ist einer der Gründe, weshalb ein Server sehr praktisch ist. Mit diesem ist dieses
        Problem gelöst. Server sind aber teuer und aufwändig. Wir müssten das ganze Multiplayer System unseres Spieles anpassen.
    \item \textbf{Helden} \\
        Eine Auswahl von Helden, mit unterschiedlichem Schaden, Leben und Effekten, würde das Spiel nochmals spannender machen. Auch sollten gewisse Truppen auf bestimmte Helden limitiert sein. 
    \item \textbf{Design} \\
        Wenn die Zeit vorhanden ist, kann das Design immer verbessert werden. Hier gehts es aber um den Feinschliff, z.B. mehr Hintergründe, und mehr Dinge selbst zu designen.
    \item \textbf{Tutorial} \\
        Eine Anleitung für Anfänger wäre sehr schön. Sie soll neuen Spielern das Anfangen erleichtern. Es sollte aber auch überspringbar sein,
        damit alte Spieler, es nicht nochmals spielen müssen. Es soll nicht lange sein und nicht schwierig. Dennoch soll es alle Mechaniken des Spiels
        erklären, am besten Anhand von Gameplay.
    \item \textbf{Truppen}
        Weitere Truppen können jederzeit erstellt werden. Hier ist kein Limit gesetzt. Leben, Schaden, Darstellung, Effekt und vieles mehr kann angepasst werden.
    \item Effekte
    \begin{itemize}
        \item \textbf{Wiederbelebung:}
            Die Truppe wird wiederbelebt, sofort oder mit einer Verzögerung am Startpunkt.
        \item \textbf{Rüstung:}
            Die Truppe hat zusätzlichen zu den Leben Rüstung. Die Rüstung wird zuerst abgezogen und hat im Gegensatz zum Leben keine Limite.
        \item \textbf{Aura:}
            Die Truppe fügt gegnerischen Truppen in einem gewissen Radius permanent Schaden zu.
    \end{itemize}
    \item \textbf{Mehrsprachig} \\
        Das Spiel soll in Englisch, Deutsch und Französisch spielbar sein.
\end{itemize}

\subsection*{Verworfen}
\begin{itemize}
    \item \textbf{Online Multiplayer} \\
        Besser als Multiplayer limitiert auf dasselbe Netz, ist Multiplayer, der weltweit verfügbar ist. Jedoch ist von einem Computer auf einen anderen zu
        verbinden dank Firewall von Router und Computer, nahezu unmöglich. Dies ist einer der Gründe, weshalb ein Server sehr praktisch ist. Mit diesem ist dieses
        Problem gelöst. Server sind aber teuer und aufwändig. Wir müssten das ganze Multiplayer System unseres Spieles anpassen.
    \item \textbf{Shop} \\
        Nicht alle Karten sind von Beginn an freigeben, sondern müssen freigespielt werden.
        So kann zum Beispiel eine Ingame Währung erspielt werden und damit im Shop Karten oder sonstige Dinge gekauft werden.
        Der Shop kann mit Zahlungsmethoden ausgestattet werden und so zur Monetariesierung beitragen.
        Jedoch ist ein Shop ohne Anti-Cheat und/oder Server die perfekte Angriffsfläche für Cheater und hat somit nicht viel Sinn.
    \item \textbf{Kampagne} \\
        Singleplayer gegen bestimmte vorprogrammierte Gegner. Sie sollen immer schwieriger werden und bestimmte Herausforderungen mit sich bringen.
        Bei Vollendung werden Belohnungen verteilt.
    \item \textbf{Anti-Cheat} \\
        Dies ist für uns ohne Erfahrung in diesem Bereich und einem lokal berechnetet Spiel ein Ding der Unmöglichkeit.
        Wir haben keine Erfahrung in diesem Bereich und es ist ein extrem komplexes Thema.
    \item \textbf{Errungenschaften} \\
        Eine Übersicht und die Möglichkeit die Errungenschaften zu erfüllen.
        Gegebenenfalls Belohnungen verteilen, wie zum Beispiel bestimmte Karten freischalten.
        Diese Erweiterung ist keines Falls notwendig und ein absolutes nice-to-have Feature.
\end{itemize}

% \includegraphics[height=10cm, width=\textwidth]{resources/mockups/mockup-login}
\chapter{Ideen}

\section{Skizzen}

\includegraphics*[width=7cm]{resources/SK_startpage.png} \quad \includegraphics*[width=7cm]{resources/SK_auswahl.png}\\
\textit{Startseite} \qquad \qquad \qquad \qquad \qquad \qquad \qquad \quad \textit{Auswählen von Karten und Helden}
\\
\begin{center}
    \includegraphics*[width=14.5cm]{resources/sk_gamemain.png}\\
\end{center}
\textit{Spielszene mit UI}

\begin{center}
    \includegraphics*[width=12cm]{resources/sk_dragndrop.png}
\end{center}
\qquad \quad \enspace \textit{Mechanik der Karten / Spawnen von Truppen}

\section{Mindmap}
\begin{center}
    \includegraphics*[width=14.5cm]{resources/sk_mindmap1.png}
\end{center}


\section{Überlegungen zum Spielprinzip}



% \includegraphics[width=\textwidth]{resources/diagrams/domain-model}

\chapter{Risiko}
Wir haben die Risiken eingeteilt in verschiedene Gefahrenstuffen.
Die Einstuffung geschah basierend auf Wahrscheinlichkeit, Wie schlimm ist der Worstcase und wie gut kann das Risiko reduziert werden.
Jedes Risiko ist dann unterteilt in:
\begin{enumerate}
    \item Risiko (Wahrscheinlichkeit und Schaden)
    \begin{enumerate}
        \item Reduktionsstrategie: Strategie um die Wahrscheinlichkeit und/oder Schaden zu reduzieren.
        \item Reduziertes Risiko: Risiko nach der Umsetzung der Reduktionsstrategie.
        \item Tatsächliches Outcome: Das tatsächliche Endresultat
    \end{enumerate}
\end{enumerate}

\section{Hoch}
\begin{enumerate}
    \item Vergange Maturaarbeiten, welche ein Videospiel als Ziel hatten, sind gescheiert (Wahrscheinlich und Kritisch)
    \begin{enumerate}
        \item Reduktionsstrategie: Sich des Risiko bewusst sein und bereit sein viel Zeit und Arbeit zu investieren
        \item Reduziertes Risiko: Das Risiko ist reduziert immernoch sehr hoch.
            Wir haben uns beide an Multiplayer fest gefressen und werden es höchst wahrscheinlich Einewegs umsetzen
        \item Tatsächliches Outcome: Wie erwartet ist die Maturaarbeit nach 250h immernoch weit entfernt von vollständig.
            Dennoch ist das Ziel erreicht von einem funktionierendem Videospiel.
    \end{enumerate}

    \item Beide Teammitglieder haben keine Ahnung von der Entwicklung einer Multiplayer-Funktion (Wahrscheinlich und Kritisch)
    \begin{enumerate}
        \item Reduktionsstrategie: Andere Spielideen erarbeiten, die ohne Multiplayer funktionieren.
        \item Reduziertes Risiko: Das Risiko ist reduziert immernoch sehr hoch.
              Wir haben uns beide an Multiplayer fest gefressen und werden es höchst wahrscheinlich Einewegs umsetzen
        \item Tatsächliches Outcome: Multiplayer ist mit Abstand der grösste Zeitfresser.
              Der Multiplayer alleine hat über 70h Arbeit, also fast ein Drittel unserer Arbeit ausgemacht.
    \end{enumerate}
\end{enumerate}

\section{Mässig}
\begin{enumerate}
    \item Beide Teammitglieder haben keine Ahnung von der Entwicklung einer Multiplayer-Funktion (Möglich und Kritisch)
    \begin{enumerate}
    \item Reduktionsstrategie: Andere Spielideen erarbeiten, die ohne Multiplayer funktionieren.
    \item Reduziertes Risiko: Das Risiko ist reduziert immernoch sehr hoch.
          Wir haben uns beide an Multiplayer fest gefressen und werden es höchst wahrscheinlich Einewegs umsetzen
    \item Tatsächliches Outcome: Multiplayer ist mit Abstand der grösste Zeitfresser.
          Der Multiplayer alleine hat über 70h Arbeit, also fast ein Drittel unserer Arbeit ausgemacht.
    \end{enumerate}
%     \begin{enumerate}
%         \item Mitigation strategies: apply cone of uncertainty, apply definition of ready to ensure planning quality
%         \item Mitigated risk: Low
%     \end{enumerate}
% 
%     \item Poor risk management (likely and critical)  
%     \begin{enumerate}
%         \item Mitigation strategies: likelihood calculation, risk mitigation plans and monitoring of risks every planning
%         \item Mitigated risk: Low
%     \end{enumerate}
% 
%     \item Project reviewer's expectations are not aligned with project (possible and critical) 
%     \begin{enumerate}
%         \item Mitigation strategies: obtain frequent approval and acknowledgement (naturally happens for us with review meetings)
%         \item Mitigated risk: None
%     \end{enumerate}
% 
%     \item Unexpected absence of team member (unlikely and catastrophic) 
%     \begin{enumerate}
%         \item Mitigation strategies: Code changes need to be pushed on a daily basis, stories could at any point be taken over by another team member
%         \item Mitigated risk: Medium
%     \end{enumerate}
\end{enumerate}

\section{Tief}
\begin{enumerate}
    \item Reduktionsstrategie: Andere Spielideen erarbeiten, die ohne Multiplayer funktionieren.
    \item Reduziertes Risiko: Das Risiko ist reduziert immernoch sehr hoch.
          Wir haben uns beide an Multiplayer fest gefressen und werden es höchst wahrscheinlich Einewegs umsetzen
    \item Tatsächliches Outcome: Multiplayer ist mit Abstand der grösste Zeitfresser.
          Der Multiplayer alleine hat über 70h Arbeit, also fast ein Drittel unserer Arbeit ausgemacht.
%     \begin{enumerate}
%         \item Mitigation strategies: code reviews, clear coding standards, apply definition of done
%         \item Mitigated risk: None
%     \end{enumerate}
% 
%     \item Lack of ownership (possible and marginal) 
%     \begin{enumerate}
%         \item Mitigation strategies: setting clear responsibilities for roles
%         \item Mitigated risk: None
%     \end{enumerate}
% 
%     \item Losing sight of documentation tasks (possible and marginal) 
%     \begin{enumerate}
%         \item Mitigation strategies: documentation strategy, documentation part of definition of done
%         \item Mitigated risk: Low
%     \end{enumerate}
% 
%     \item Failure of hardware like personal devices, OST GitLab, Jira, hosted environment (rare, catastrophic) 
%     \begin{enumerate}
%         \item Mitigation strategies: Code changes need to be pushed on a daily basis
%         \item Mitigated risk: Low
%     \end{enumerate}
\end{enumerate}

\part{Produkt}
Wir konnten fast all unsere Anforderungen erfüllen.

\section{KZO}
% \includegraphics[height=18cm]{resources/diagrams/use-case}
\begin{enumerate}
    \item \textbf{Zeit} \\
        Geplant war ungefähr eine Lektion pro Woche. Wir haben die geschätzten 50 Stunden um einen Faktor von sechs überschritten. Unser gemessener Zeitaufwand, während aktivem Programmieren,
        beträgt über 330 Stunden. Da sind die Stunden, in denen wir uns in der Freizeit, und mit Herrn Hunziker getroffen haben, gar nicht notiert.
    \item \textbf{Begleitperson} \\
        Wir hatten zum Glück kein Problem eine Begleitperson zu finden. Martin Hunziker war gerne bereit unsere Arbeit im Blick zu behalten.
    \item \textbf{Plagiat} \\
        Zwar sind einige Grafiken aus dem Internet, jedoch sind alle Nutzungsfrei. Auch sonst haben wir keinen Code geklaut und verwendet. 
\end{enumerate}

\subsection*{Notwendig}
\begin{itemize}
    \item \textbf{System} \\
        Unser Spiel funktioniert sowohl auf macOS als auch auf Windows. Dies ausserdem mit einer FPS von 300, bis teilweise 500, bei den meisten modernen Geräten
    \item \textbf{Benutzeroberfläche} \\
    \begin {itemize}
        \item \textbf{Startpage} \\
            \\
            \includegraphics[height=7cm]{resources/laneclash.png}\\
            Unsere Startseite hat einen ''Spielen''-Knopf, einen ''Einstellungen''-Knopf, einen ''Credits''-Knopf und einen ''Verlassen''-Knopf. Unsere vereinzelten Tester bewerteten die Startseite
        als intuitiv verständlich. Bedienbar mit der Maus, hatten die Tester keine Probleme sich zurechtzufinden.
        \item \textbf{Credits}\\
            \\
            \includegraphics*[width = 7cm]{resources/credits.png}\\
            Es gibt die Möglichkeit die Credits vorzeitig zu Verlassen. Dies geschieht durch das Drücken des Knopfes mit dem Haus als Icon. Die Tester fanden dies auch sehr intuitiv.
        \item \textbf{Settings}\\
            \\
            \includegraphics*[height=7cm]{resources/setting.png}\\
            Man kann hier die Auflösung ändern und auswählen, ob das Spiel im Vollbild dargestellt werden soll. Auch gibt es hier die Möglichkeit die Musik lauter, leiser zu stellen
            oder zu muten. Wenn man Einstellungen geändert hat, muss man dies bestätigen. \\
            \includegraphics*[height=3cm]{resources/resolution.png}\\
            Wenn man die Auflösung ändert und dies bestätigt, werden die Änderungen angewendet und man hat zehn Sekunden Zeit erneut zu bestätigen. Sonst werden die Änderungen zurückgesetzt.
            Dies wurde implementiert, um eventuellen Problemen vorzubeugen. 
        \item \textbf{Server Management}\\
            \\
            \includegraphics*[height = 3cm]{resources/server.png}\\
            Nach Interaktion mit dem ''Start''-Knopf gibt es die Möglichkeit einen Server zu Hosten oder einer bereits existierenden Lobby als Client beizutreten. Ausserdem kann man
            automatisch nach bereits existierenden Servern suchen.\\
            \\
            \includegraphics*[width=6cm]{resources/discovered.png}\\
            Den gewollten Server kann man mit einem einfachen Linksklick auswählen.
        \item \textbf{Lobby Management}\\
            \\
            \includegraphics*[width = 5cm]{resources/lobby.png} \includegraphics*[width=5cm]{resources/lobbyc.png}\\
            Player (1) ist dabei immer der Host, und Player (2) immer der Client. Das eigentliche Spiel beginnt erst, sobald beide Spieler bereit, also ''ready'', sind. 
            Dieser Status ''ready'' wird dabei immer aktualisiert dargestellt. Der Host hat ausserdem die Möglichkeit den Spieler (2) aus der Lobby zu entfernen. Wir wollten zuerst
            eine Möglichkeit für einen ''Schlüssel'' hinzufügen, allerdings stellte dies sich als überflüssig dar, da alle Spieler unter einem Dach sitzen.
        \item \textbf{InGame-UI}\\
            \\
            \includegraphics*[height=2cm]{resources/returnlobby.png} \includegraphics*[height=2cm]{resources/stopclient.png}\\
            Der Host hat im Spiel die Möglichkeit, den Server zurück zur Lobby zu schicken, und das Spiel somit neuzustarten. Er kann auch direkt aufhören zu hosten. 
            Auch der Client hat durchgehend die Möglichkeit den Server zu verlassen.\\
            \\
            \includegraphics*[height=2cm]{resources/pause.png}\\
            Im Pausemenu kann man ausserdem auf die Schnelle die Musiklautstärke und die Stärke des Schneefalls ändern.


    \end {itemize}

    \item \textbf{Lokaler Multiplayer} \\
        Das Spiel kann im lokalen Netz gespielt werden. Zum Testen auch auf demselben Gerät. Als Protokoll haben wir TCP gewählt. Dies, da TCP viel Zuverlässiger ist.
        Folgende vier Punkte waren entscheidend in der Wahl von TCP über UDP:
        \begin{itemize}
            \item \textbf{Zuverlässigkeit}\\
                Keine Probleme mit verloren gegangenen Paketen. Ging ein Paket verloren, sendet TCP es einfach erneut. Entweder werden alle Daten erfolgreich übermittelt,
                oder man bekommt einen Error.
            \item \textbf{Sequenziert}\\
                TCP stellt sicher, dass jede Nachricht in der gleichen Reihenfolge eintrifft, in der sie gesendet wurde. Wenn man ''a'' dann ''b'' sendet, erhält man auf der
                anderen Seite garantiert auch zuerst ''a'' dann ''b''.
            \item \textbf{Verbindungsorientiert}\\
                TCP hat das Konzept einer Verbindung. Eine Verbindung bleibt so lange offen, bis entweder der Client oder der Server sich dazu entscheiden, sie zu schliessen.
                Anschliessend werden sowohl Client als auch Host benachrichtigt, dass die Verbindung beendet wurde.
            \item \textbf{Überlastungskontrolle} \\
                Wenn ein Server überlastet wird, drosselt TCP selbstständig die Daten, um einen Zusammenbruch durch Überlastung zu verhindern.


        \end{itemize}
    \item \textbf{Einstellungen} \\
        Auflösung, Vollbild, Lautstärke und Stärke des Schneefalls können eingestellt werden.
    \item \textbf{Kamera} \\
        Die Kamera ist beweglich und das ganze Schlachtfeld ist sichtbar. Dabei kann man auch zoomen. Kontrollierbar ist dies mit den Pfeiltasten.
    \item \textbf{Truppen} \\
        IMPORTANT\\
        IMPORTANT\\
        IMPORTANT\\
        IMPORTANT\\
        IMPORTANT\\
        IMPORTANT\\
        IMPORTANT\\
        IMPORTANT\\
        IMPORTANT\\
        IMPORTANT\\
        IMPORTANT\\
        IMPORTANT\\
        IMPORTANT\\
        IMPORTANT\\
        IMPORTANT\\
        IMPORTANT\\
        IMPORTANT\\
    \item \textbf{Gewinnmöglichkeit} \\
        Um zu Gewinnen, muss eine eigene Truppe die Portale des Gegners durchschreiten.
    \item \textbf{Lanes} \\
        Unser Spiel hat 4 sogenannten Linien (''Lanes''). Diese unterschiedlichen Linien in der Tiefe verschoben.
    \item \textbf{Synchronisation} \\
        Unser Spiel Synchronisiert alle Daten in Echtzeit. Dadurch sehen alle Spieler immer das Gleiche. Wird der Client aus einem unbekannten Grund disconnected kann er versuchen
        sich erneut zu verbinden und der Server wird dann direkt alle Daten synchronisieren. Auch hier hilft und das gewählte Protokoll TCP.
    \item \textbf{Crossplay}
        Die Möglichkeit sich mit einem Windows Rechner und einem macOS Rechner zu verbinden und zusammenzuspielen.
\end{itemize}

\subsection*{Anvisiert}
\begin{itemize}
    \item \textbf{Helden} \\
        Beide Spieler haben eine Art Truppe, welche ihre Siegeslinie beschützt. Sie haben einen Racheeffekt, welcher die
        Lane ausradiert, damit das Spiel nicht sofort vorbei ist.
    \item \textbf{Design} \\
        Das Spiel sollte vom Aussehen her was hergeben. Es sollte nicht wie ein Prototyp, sondern wie ein 
        vollendetes Spiel aussehen. 
    \item \textbf{Deck} \\
        Zu Beginn Spieler könne ihr eigenes Deck aus einer Auswahl von Karten zusammenstellen. Im Spiel werden davon per Zufall Karten gezogen.
    \item \textbf{Bot} \\
        Ein sehr simpler Algorithmus um alleine gegen den Computer zu spielen auf den Schwierigkeitsstufen \textit{'Einfach', 'Mittel', 'Schwierig'}.
        Es sollen rein zufällig Truppen geschickt werden, bei höherer Schwierigkeit mehr.
    \item \textbf{Truppen}
    \begin{itemize}
        \item \textbf{Nahkampf}
            Eine Truppe mit wenig Reichweite.
        \item \textbf{Fernkampf}
            Eine Truppe die auf Reichweite angreift. Sie scheisst Projektive, zum Beispiel ein Bogenschütze,
            der mit Pfeilen schiesst.
        \item \textbf{Suizid}
            Eine Truppe die bei Berührung mit einem Gegner stirbt und einen Effekt auslöst.
    \end{itemize}
    \item \textbf{Effekte}
    \begin{itemize}
        \item \textbf{Gift:}
            Zeitlich limitierter und wiederholender Schadenseffekt. Eine visuelle Markierung soll vorhanden sein
            und das gleiche Gift, also der gleichen Truppe, soll nur einmal auf jemanden sein.
        \item \textbf{Dorn:}
            Truppen, welche Charaktere mit Dorn attackieren, erhalten schaden. Soll nach Auswahl auf Nahkämpfer,
            Fernkämpfer oder beides wirken.
        \item \textbf{Rache:}
            Truppen mit Rache haben einen Effekt nach ihrem Tod. Zum Beispiel eine Truppe beschwören oder Schaden verursachen.
    \end{itemize}
\end{itemize}

\subsection*{Möglichkeiten}
\begin{itemize}
    \item \textbf{Zauber} \\
        Als Alternative sollen nicht alle Karten einfach eine Truppe herbeirufen. Gewisse Karten sollten nur einen Effekt auslösen.
        Beispiele für Zauber:
            "Heile deine Helden um 5 leben"
            "Gib deinen Helden 5 Rüstung"
            "Verursache allen gegnerischen Truppen 5 Schaden"
            "Ziehe 3 Karten"
    \item \textbf{Monetarisierung} \\
    Hierfür sind uns drei Möglichkeiten eingefallen, hier jeweils die Vor- und Nachteile:
    \begin{enumerate}
        \item \textbf{Kaufbare Gegenstände}
        \begin{itemize}
            \item[+] Vielseitige und Nachhaltige Monetarisierung. Neue Features, bedeuten auch neue Geldmöglichkeit. Die Inhalte sollten
                     auch erspielbar sein, somit kann bezahlen zwar zu Vorteilen führen, jedoch mit viel Spielen trotzdem erreichbar sein.
            \item[-] Pay-to-Win
        \end{itemize}
        \item \textbf{Skins}
        \begin{itemize}
            \item[+] Weit verbreitet und sehr beliebt bei Spielern. Gibt keinem Spieler Vorteile
            \item[-] Wenig Nutzen und sehr aufwändig. Neue Designs zu erstellen, wäre bei uns nicht so sinnvoll.
                     Wir haben noch sehr viel Features zu implementieren und sind nicht Designer. Für uns braucht es sehr
                     viel Zeit eine brauchbare Darstellung zu erstellen und der Mehrwert, welcher dem Spiel beigetragen wird, ist marginal.
        \end{itemize}
        \item \textbf{Kostenpflichtiges Spiel}
        \begin{itemize}
            \item[+] Einmalige Bezahlung, was für viele Nutzer lukrativer ist. Jedoch gilt dies vor allem bei Singleplayer Spielen.
            \item[-] Wir nehmen an, dass wenige Leute bereit wären, Geld für unser Spiel zu bezahlen.
                    Dafür wird es nicht genug ausgereift sein.
                    Auch ist diese Methodik nicht nachhaltig und führt nur zu einer einmaligen Geldspritze.
                    Viele grössere Spiele führen deshalb später
                    DLCs ein, um das Spiel zu erweitern.
                    Jedoch ist dies bei einem Multiplayer Spiel Pay-to-Win.
                    Abschliessen müssten wir höchstwahrscheinlich selbst Geld vorauswerfen, um unser Spiel anbieten zu können, z.B. auf Steam.
        \end{itemize}
        
    \end{enumerate}
    \item \textbf{Helden} \\
        Eine Auswahl von Helden, mit unterschiedlichem Schaden, Leben und Effekten, würde das Spiel nochmals spannender machen. Auch sollten gewisse Truppen auf bestimmte Helden limitiert sein. 
    \item \textbf{Design} \\
        Wenn die Zeit vorhanden ist, kann das Design immer verbessert werden. Hier gehts es aber um den Feinschliff, z.B. mehr Hintergründe, und mehr Dinge selbst zu designen.
    \item \textbf{Tutorial} \\
        Eine Anleitung für Anfänger wäre sehr schön. Sie soll neuen Spielern das Anfangen erleichtern. Es sollte aber auch überspringbar sein,
        damit alte Spieler, es nicht nochmals spielen müssen. Es soll nicht lange sein und nicht schwierig. Dennoch soll es alle Mechaniken des Spiels
        erklären, am besten Anhand von Gameplay.
    \item \textbf{Truppen}
        Weitere Truppen können jederzeit erstellt werden. Hier ist kein Limit gesetzt. Leben, Schaden, Darstellung, Effekt und vieles mehr kann angepasst werden.
    \item Effekte
    \begin{itemize}
        \item \textbf{Wiederbelebung:}
            Die Truppe wird wiederbelebt, sofort an Ort und Stelle oder mit einer Verzögerung am Startpunkt.
        \item \textbf{Rüstung:}
            Die Truppe hat zusätzlichen zu den Leben auch Rüstung. Die Rüstung wird zuerst abgezogen und hat im Gegensatz zum Leben keine Limite.
        \item \textbf{Aura:}
            Die Truppe fügt gegnerischen Truppen in einem gewissen Radius permanent Schaden zu.
    \end{itemize}
    \item \textbf{Mehrsprachig} \\
        Das Spiel soll in Englisch, Deutsch und Französisch spielbar sein.
\end{itemize}

\subsection*{Verworfen}
\begin{itemize}
    \item \textbf{Online Multiplayer} \\
        Besser als ein Multiplayer, der limitiert auf dasselbe Netz ist, ist ein Multiplayer, der weltweit verfügbar ist. Jedoch ist von einem Computer auf einen anderen zu
        verbinden dank der Firewall von Router und Computer, nahezu unmöglich. Es würden sich viele weitere Probleme ergeben, wie zum Beispiel Port-Forwarding. Dies ist einer der Gründe, weshalb ein Server sehr praktisch ist. Mit diesem ist dieses
        Problem gelöst. Server sind aber teuer und aufwändig. Wir müssten allerdings das ganze Multiplayer System unseres Spieles anpassen, was einen kompletten Recode zur Folge hätte.
    \item \textbf{Shop} \\
        Nicht alle Karten sind von Beginn an freigeben, sondern müssen freigespielt werden.
        So kann zum Beispiel eine Ingame Währung erspielt werden und damit im Shop Karten oder sonstige Dinge gekauft werden.
        Der Shop kann mit Zahlungsmethoden ausgestattet werden und so zur Monetarisierung beitragen.
        Jedoch ist ein Shop ohne Anti-Cheat und/oder Server die perfekte Angriffsfläche für Cheater und hat somit nicht viel Sinn.
    \item \textbf{Kampagne} \\
        Singleplayer gegen bestimmte vorprogrammierte Gegner. Sie sollen immer schwieriger werden und bestimmte Herausforderungen mit sich bringen.
        Bei Vollendung werden Belohnungen verteilt.
    \item \textbf{Anti-Cheat} \\
        Dies ist für uns ohne Erfahrung in diesem Bereich und einem lokal berechnetet Spiel ein Ding der Unmöglichkeit.
        Wir haben keine Erfahrung in diesem Bereich und es ist ein extrem komplexes Thema.
    \item \textbf{Errungenschaften} \\
        Eine Übersicht und die Möglichkeit die Errungenschaften zu erfüllen.
        Gegebenenfalls Belohnungen verteilen, wie zum Beispiel bestimmte Karten freischalten.
        Diese Erweiterung ist keines Falls notwendig und ein absolutes nice-to-have Feature.
\end{itemize}
\chapter{Architektur}
Die folgenden UML-Diagramme sind nicht vollständig.
Wir versuchen einen kleinen aber dennoch guten Einblick in unsere Architektur zu liefern.
Dieser Abschnitt sicher spannend für technikinteressierte Leser. Man muss jedoch kein Informatiker sein, um dieses Kapitel zu verstehen.

\section{Klassendiagramm}
Folgend ist der Klassenaufbau unseres Spieles zu sehen.
Das rote (S) steht für Singleton.
Dies sind Klassen, die im ganzen Spiel nur genau einmal existieren können.
Die erste Graphik zeigt den momentanen Stand, wobei mit der Entwicklung des Deckbauers die neue Klasse "DeckManager" hinzukommen wird. 
\begin{figure}[H]
    \centering
    \includegraphics[width=13cm]{resources/Singletons.png} \\
    \caption{momentaner Aufbau}
\end{figure}

\begin{figure}[H]
    \centering
    \includegraphics[width=13cm]{resources/Singletons 2.png} \\
    \caption{zukünftiger Aufbau}
\end{figure}

\section{Sequenzdiagramme}
\subsection{Nahkampftruppe}
\begin{figure}[H]
    \centering
    \includegraphics[width=15cm]{resources/MeleeAttacks.png}\\
    \caption{Abfolge, wenn eine Truppe in Reichweite einer Nahkampftruppe kommt}
\end{figure}

\subsection{Suizidtruppe}
\begin{figure}[H]
    \centering
    \includegraphics[width=15cm]{resources/SuicideAttacks.png}\\
    \caption{Suizidtruppe mit ähnlichen Prinzipien}
\end{figure}



\subsection{Fernkampftruppe}
\begin{figure}[H]
    \centering
    \includegraphics[width=15cm]{resources/RangedAttacks.png}\\
    \caption{Reaktion einer Fernkampftruppe auf neue Gegner}
\end{figure}
\begin{figure}[H]
    \centering
    \includegraphics[width=15cm]{resources/Projectile.png}\\
    \caption{Gegner stirbt bevor das Projektion Schaden anrichtet}
\end{figure}


\subsection{Gift}
\begin{figure}[H]
    \centering
    \includegraphics[width=15cm]{resources/Poison.png} \\
    \caption{Reaktion einer Truppe die den Effekt Gift besitzt}
\end{figure}



    \section{GitHub file system / source / version control}
\url{https://www.youtube.com/watch?v=IQT37uwpcg4}
\input{02_produkt/03_feedback}
\input{02_produkt/04_Zukunft}

\part{Schwierigkeiten}
\chapter{Technologien}

\section{Unity}
Eine Lauf- und Entwicklungsumgebung für Spiele.
Sehr beliebt für Indie-Developer.
Unity ist ein mächtiges und sehr hilfreiches Programm bei der Entwicklung Videospielen, einige Vorteile:
\begin{itemize}
    \item Physik
    \begin{itemize}
        \item Schwerkraft
        \item Reibung
        \item Beschleunigung und Bremsen
    \end{itemize}
    \item Visuell
    \begin{itemize}
        \item Schwerkraft
    \end{itemize}
\end{itemize}

\section{Github}


\section{LaTex}


\chapter{Team Management}

\section{Anfänge}
\begin{itemize}
    \item Bereits Mitte Januar 2022 fragte Marc Elia an, ob sie nicht zusammen die Maturitätsarbeit schreiben wollten. Wir hatten beide vor, als Maturaarbeit etwas zu programmieren. Also vereinigten wird unsere Kräfte.
    \item Die nächste Frage war, was wir genau als Projekt programmieren wollten. Es stellte sich sehr schnell heraus, dass sich Marcs Grundidee (ein Videospiel), perfekt eignen würde. Man kann seiner Fantasie bei der Erstellung eines Videospiels
    freien Lauf lassen.
    \item Marc brachte am Anfang direkt den Vorschlag, dass wir unsere Zeit mit Jira tracken würden. Er setzte zudem Github und Unity für unseren ersten Gebrauch auf. 
    \item Elia konnte sich sehr schnell in dieser unbekannten Umgebung zurechtfinden. So entstand ein begeisterungsfähiges gleichberechtigtes Duo.
\end{itemize}

\section{Betreuungsperson}
\begin{itemize}
    \item Uns wurde vor Beginn der Arbeit mehrfach gesagt, dass einige Schüler Probleme haben, eine Betreuungsperson zu finden. Um sicherzugehen, dass dies uns nicht geschehen würde, suchten wir bereits sehr früh nach einer Lehrperson.
    \item Elias erster Vorschlag war Herr Albert Kern. Er hat ihn bereits vor längerer Zeit provisorisch für eine individuelle Betreuung angefragt. Was lag näher, als ihn nun auch für eine Gruppenarbeit anzufragen? Wie es der Zufall wollte, begegneten sich 
    Herr Kern und Elia Ende Februar zufälligerweise auf dem Gang und tauschten einige Worte aus. Herr Kern gab allerdings zu, dass er keine Erfahrung mit der Entwicklung von Videospielen hatte und schlug Elia Martin Hunziker vor.
    Dieser sei ein begnadeter Videospiel-Fanatiker und wahrscheinlich besser für die Betreuung geeignet. 
    \item Am dritten März schrieben wir eine kurze Anfrage an Herr Hunziker per E-Mail. Herr Kern hatte zuvor im Gespräch mit Elia erwähnt, dass er dabei gerne ein gutes Wort bei Herr Hunziker einlegen würde. Ob dies tatsächlich 
    stattfand, entzieht sich unserem Wissen. Allerdings sassen wir 4 Tage später, am siebten März bereits bei Herr Hunziker im Büro und besprachen mit ihm unsere Idee. 
    \item Eine Woche später stand seine Unterschrift auf unseren beiden Blättern bezüglich der Maturitätsarbeit. Allerdings hat selbst Herr Hunziker gestanden, dass er uns bei den meisten Problemen nicht helfen können, weil er
    keine grosse Erfahrung hat.
\end{itemize}

\section{Kommunikation}
\begin{itemize}
    \item Unsere Kommunikation fand vor allem über WhatsApp statt. Dies geschah allerdings nicht nur in Textform, sondern des Öfteren auch in Telefonaten. Zudem konnte man auf Jira nachschauen, auf welche Aufgabe der Partner
    seine Zeit getracked hat. 
    \item Wir müssen allerdings zugeben, dass die Kommunikation teilweise sehr gering vorhanden war. Teilweise hatte man keine Ahnung, woran genau der Partner arbeitete. Wir wussten zwar das Ungefähre, allerdings manchmal keine Einzelheiten.
    \item Die Kommunikation zwischen Schüler und Lehrer war unserer Meinung nach mit sechs Treffen genügend gut. Zudem trafen wir uns dreimal neben der Schule, um unsere Arbeit zu organisieren.
\end{itemize}




\chapter{Time Management}

\section{Roadmap}
\includegraphics[height=7cm]{resources/Roadmap.png}\\

\section{Time-Tracking}
\includegraphics*[width=15cm]{resources/graph.png}
HIER NOCH ENDZEITEN IN LISTE SCHREIBEN

\part{Reflexion}
\chapter{Team}

\section{Rückblick}

\subsection*{Kommunikation}
Wir verstanden uns während der ganzen Arbeit sehr gut.
Es kam nie zu zwischenmenschlichen Problemen. Jedoch haben wir viel Potenzial der Teamarbeit nicht ausgeschöpft.
Eigentlich war es mehr zwei Schüler die am gleichen Projekt arbeiten.
Natürlich haben wir die Idee zusammen erarbeitet, haben uns getroffen und hatten das selbe Ziel, jedoch wars das.
Wir hatten kein Schutzmechanismus für Fehler einer Person.
So haben wir einmal etwa 5h verloren wegen einem kleinem Schreibfehler.
Wir haben uns relativ wenig getroffen, einmal für die Idee, einmal für das Timemanagemnt und zweimal für sonstiges.
Wir haben viel Code im Spiel, welcher nur einer von uns versteht.
Wir waren beide häufig unsicher wie wir Dinge dokumentieren und was alles als investierte Zeit giltet.

\subsection*{Agilität}
Ein anderer Punkt ist die Steifheit unseres Projekts.
Ein Ziel und eine Vision ist wichtig, jedoch sollte sie jederzeit angepasst werden können.
Wir haben zu Begin sehr viel getüftelt, das auch schon an Dingen die zu Beginn völlig irelevant sind.
So haben wir z.B. 8 Effekte erfunden, bevor wir überhaupt ein Spiel hatten.
In der Informatik nennt sich das "das Wasserfall-Prinzip" und wird als schlecht erachtet.
In der modernen Entwicklung fokusiert man sich auf das agile entwicklen.

\subsection*{Priorisierung}
Wir hatten eine schlechte Priorisierung.
So hat unser Spiel z.B. ein schönes UI und ein funktionsfähiges Pausenmenu, aber es fehlen einige Ingame Features, wie z.B. das Deck bauen.

\subsection*{Manpower}
Uns sind einfach Grenzen gesetzt zuzweit.
Wir sind uns einig, eine dritte Person hätte geholfen.
Entweder eine gestalterische Person, welche die ganze Gestaltung übenrommen hätte, oder ein weiter Entwickler.
Wir denken auch, drei Leute ist immernoch ein kleines Team und es sollte noch zu keinen grossen Team management Problemen führen.

\subsection*{Dokumentation}
Wir haben die Dokumentation zu weit nach hinten geschoben.
Anfangs hatten wir eine sehr genaue und teils Maturaarbeit reife Dokumentation, jedoch ging dies später verloren.
So sind viele Zeitaufwandeinträge ohne Kommentar oder mit sehr wenig Erläuterung.

\section{Nächstes Mal}

\subsection*{Treffen}
Mehr Treffen, Calls und Klarheit.
Bei einer weiteren Arbeit würdne wir uns ein mal monatlich treffen um einfach über den Stand der Dinge zu reden.
Wo genau sind wir jetzt?
Wer hat was in nächster Zeit vor?
Gibt es Features die wir vergessen haben?
Gibt es neu Ideen für Features?
Sind usnere Skizzen noch Zeit getreu?
Braucht es Neue?
So soll das Spiel Schritt für Schritt wachsen

\subsection*{Vier-Augen-Prinzip}
Üblich in der Entwicklung mit GitHub sind sogenannte Pullrequests.
Falls ein Mitglied änderen vornimmt, macht er eine Anfrage für seine Anpassungen.
Diese müssen dann von jemand anderem genehmigt werden.
So sollen Fehler reduziert und schlussendlich Zeit gesparrt werden.
Dieses Prinzip würden wir in einer weiteren Arbeit auch anwenden.

\subsection*{Priorisierung}
Dieser Punkt ist schwierig anzupassen.
Mit mehr Treffen, Besprechung und Planun sollte dieses Problem sicher minimiert sein.

\subsection*{Verschwendete Zeit}
Zwar handelt sich nur um wenige Stunden, dennoch haben wir viel in die Zukunft geschaut.
So macht es zum Beispiel Sinn sich zu entscheiden, dass Effekte exisiteren und sich als Proof-of-Concept einen auszudenken, jedoch nicht acht.
Diese können laufend, einmal so weit gekommen, durchdacht und entwickelt werden.
\chapter{Elia}

\section{Rückblick}
Das Produkt dieser Arbeit erfreut mich sehr. Obwohl dem Prototyp unseres Videospiels noch einige Funktionen fehlen, bin ich doch grundsätzlich mit unserem Spiel zufrieden. \\
Dies ist das grösste und zeitaufwendigste Projekt, an dem ich jemals gearbeitet habe. Unser GitHub Repository enthält mehr als 40'000 Zeilen an C\# Source-Code.
Das ganze Repository ist ca. 1.1 Millionen Zeilen lang und mehrere Gigabytes gross (einschliesslich DLL's etc.). \\
Alleine hätte ich dies nie geschafft, deshalb bin ich besonders froh, dass wir dies im Team verwirklichen konnten. Die Gruppenarbeit hat mir viel Druck von den Schultern genommen, da ich mich bei Problemen immer auf Marc
verlassen konnte. Zudem führte die Zusammenarbeit zu vielen guten Ideen, die wir auch nur zusammen als Team in die Tat umsetzen konnten. Durch diese Gruppenarbeit wurde unsere Freundschaft stark gestärkt.
Ich habe zu schätzen gelernt, dass ich mich die ganze Zeit voll und ganz auf Marc verlassen konnte.\\
Obwohl von Anfang an klar war, dass unser Projekt eine sehr aufwendige Arbeit sein würde, habe ich sie doch unterschätzt. Am meisten unterschätzt habe ich allerdings den Arbeitsaufwand der schriftlichen Arbeit. sie stellte sich als ziemlich grosser Zeitschlucker heraus. \\
Ich hatte zudem unterschätzt, was mein geplantes Hardwareupgrade für immensen Einfluss auf meinen Arbeitsfluss haben würde. Der Wechsel von meinem Laptop auf einen hochmodernen Computer brachte viele Vorteile mit sich.
Die vervielfachte Leistung hatte den Vorteil, dass ich bei Kompilierung des Codes keine halbe Minute mehr warten musste. Stattdessen konnte ich zwei Sekunden später unser Spiel testen.
Diese immense Rechenleistung merkte man auch daran, dass sich mein Zimmer durch die Nutzung des Computers im Laufe eines Tages - trotz geöffnetem Fenster - um mehrere Grade erwärmte.\\
Auch wenn die Fehlersuche teilweise sehr anstrengend war, hat mir das Programmieren trotz teilweise langen Tagen und Nächten meistens Spass gemacht. Ich fand es trotzdem schade,
dass ich teilweise den ganzen Tag nur im Zimmer sass und an der Maturaarbeit programmierte. 

\section{Nächstes Mal}
Für das nächste Mal möchte ich mir zu Beginn des Semesters einen Plan erstellen, der festlegt, dass ich pro Woche mindestens eine Stunde arbeiten werde. Mit diesem Plan wäre es auf keinen Fall zu dieser sehr langen
Pause (von Mai bis August) gekommen. Wenn ich erneut die Chance hätte, dieses Projekt in einer Gruppenarbeit zu lösen, würde ich diese Chance ohne zu zögern ergreifen. Allerdings würde ich in Zukunft
unsere Organisation zwingend verbessern. Eventuell wäre ein drittes Teammitglied auch keine schlechte Idee (vgl. \autoref{subsec:neusm}).\\
Was diese Erfahrung allerdings mit meinem Plan, Informatiker zu werden, geändert hat, kann ich zu diesem Zeitpunkt nicht beantworten. Fakt ist, ich habe nun auf jeden Fall eine genauere Vorstellung,
wie das Leben eines Informatikers aufgebaut ist. Zudem habe ich eine Menge an Respekt für jeden Spielentwickler gewonnen. Ein Spiel zu entwickeln ist nicht leicht, deswegen werde ich ab nun immer meine Erfahrungen im Hinterkopf
behalten und auch immer an das Dev Team denken, wenn ich über einen Bug oder Crash fluche.
\chapter{Marc}

\section{Rückblick}
Ich bin äusserst zufrieden mit unserer Arbeit.
Allerdings war meine ursprüngliche Vorstellung unseres Spiels ausgereifter.
Dies scheiterte aber nicht an der Arbeit von Elia, mir oder uns beiden, sondern einfach an meinem Hochmut.
Ich habe vor Jahren die selbe Spielidee auf eigene Faust versucht umzusetzen und kam in relativ wenig Zeit sehr weit.
Ich dachte der Fortschritt wächst Linear.
Jedoch stiessen auch wir hier auf das Paretoprinzip: 20\% Aufwand gleich 80\% Ergebnis und die restlichen 20\% Ergebnis benötigen 80\% des Aufwands.
Der Feinschliff des UI-Aussehen, der Abschluss der schriftlichen Arbeit und ein bugfreies Spiel waren relativ zu ihrer Grösse riesen Zeitfresser. \\
Die Arbeit mit Elia war voll und ganz erfreulich.
Es gab keine Probleme und es hat mir durchgehend Spass gemacht.
Wir hatten zwar wenig konkrette Gespräche, dennoch fand man sich immer wieder im Gespräch über unsere Arbeit.
Ich bin extrem froh Elia als Partner angefragt zu haben und ich denke er war für das Spiel mindestens gleich bereichernd wie ich. \\
Gegen Ende wurde die Arbeit doch noch zu einem Stress.
Ich kann nur von Glück reden, dass Elia im Endspurt so eine riesen Arbeit geleistet hat.
Für mich war es bis kurz vor Schluss sehr schwierig wieder in den Flow zu kommen.

\section{Nächstes Mal}
Vieles würde ich genauso wiederholen.
So zum Beispiel der Start unserer Arbeit, die Arbeit an sich und die Wahl meines Partners.
Aber natürlich lief fieles nicht reibungslos.
Das grösste Problem war mangelnde Planung und Routine.
Folgende 'Systeme' sollen dem bei meinem nächsten Projekt entgegenwirken:
\begin{enumerate}
    \item Ich werde mir ein Stundenminimum pro Woche setzen.
    Dies soll nicht die angepeilten Anzahl Stunden darstellen, sondern nur ein Start und somit maximal wenige Stunde pro Woche betragen.
    Dies soll verhindern, dass ich komplett aus der Arbeit rausfalle und Schwierigkeiten habe wieder reinzukommen.
    \item Ich werde ein Reflexionsystem einführen.
    In diesem soll wöchentlich oder allenfalls monatlich reflektiert werden.
    Was lief gut? Was lief schlecht? Was kann ich besser machen?
    \item Ich werde wöchentlich Brainstormen.
    Soll es allein oder zu zweit sein, aber einmal weg vom Codezeilen schreiben.
    Dies würde dem Projekt deutlich mehr Agilität und Anpassung bringen.
    Ich bin auch sicher, dies führt zu vielen und sicher auch interessanten Ideen.
\end{enumerate}
Diese Punkte hätten mir alle dabei geholfen mir das Ziel vor Augen zuführen.
Denn genau dieses Ziel verlor ich zeitweise.
Schlussendlich hatte ich weniger Spass und bekam Schwierigkeiten mich zu motivieren. \\
Da unser Projekt nicht mit der Abgabe abgeschlossen ist, werde ich nach dieser Reflektion meinen Zielen und Plänen widmen. \\
Entgegen der unerfüllten Erwartung bin ich stolz auf uns und denke wir haben für zwei Gymischüler ein brilliantes Produkt erstellt.
Nichtdestotrotz würde ich beim nächstem mal eine dritte Person mit ins Boot holen.


% Note: The project proposal needs to be included in the following way (i.e. using `\input` and not `\include`, and with the `\chapter` declaration outside the input{}-tag), since the proposal can also be generated as a stand-alone document.
% \chapter{Initial Project Proposal}
% \input{03_project-documentation/01_project-proposal}

% Ensure all entries are printed, even if not referenced
\glsaddall
\printglossary
\newpage

\addcontentsline{toc}{chapter}{Bibliography}

% Ensure all citations are printed, even if not referenced
\nocite{*}

\bibliographystyle{alpha}
\bibliography{bibliography}

\end{document}
