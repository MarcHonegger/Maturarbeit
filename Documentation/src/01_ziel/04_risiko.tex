\chapter{Risikoanalyse}
Wir haben mögliche Risiken in verschiedene Stufen eingeteilt.
Die Einstufung geschah basierend auf der Eintrittswahrscheinlichkeit (unwahrscheinlich, wahrscheinlich, sehr wahrscheinlich) und Auswirkung (unbedenklich, kritisch). 
Um das Risiko zu managen, haben wir für jedes Risiko eine Reduktionsstrategie definiert.\\
In den folgenden Teilkapiteln dokumentieren wir die identifizierten Risiken nach folgendem Schema:\\
\\
- Risiko (Eintrittswahrscheinlichkeit und Auswirkung)
\begin{enumerate}
    \item Reduktionsstrategie: Strategie, um die Wahrscheinlichkeit und/oder die Auswirkung zu reduzieren.
    \item Reduziertes Risiko: Risiko nach der Umsetzung der Reduktionsstrategie.
    \item Tatsächliches Outcome: Das tatsächliche Endresultat
\end{enumerate}

\section{Hoch}
\begin{enumerate}
    \item Vergangene Maturitätsarbeiten, die ein Videospiel als Ziel hatten, sind gescheitert (Wahrscheinlich und Kritisch)
    \begin{enumerate}
        \item Reduktionsstrategie: Klare Anforderungsspezifikation, Planung und Aufteilung der Arbeitspakete, laufende Überprüfung der Planung/Zielerreichung, regelmässiger gegenseitiger Austausch zum Fortschritt,
        Bereitschaft den empfohlenen Zeitaufwand massiv zu überschreiten.
        \item Reduziertes Risiko: Die Eintrittswahrscheinlichkeit ist reduziert. Beim Eintreten ist der Erfolg der Arbeit allerdings immer noch gefährdet.
        \item Tatsächliches Outcome: Risiko nicht eingetreten. Die Maturitätsarbeit wurde zusammen mit funktionierendem Videospiel fertiggestellt.
    \end{enumerate}

    \item Beide Teammitglieder haben keine Ahnung von der Entwicklung eines netzwerkfähigen Multiplayer-Spiels (Wahrscheinlich und Kritisch)
    \begin{enumerate}
        \item Reduktionsstrategie: Erstellen der grundsätzlichen Multiplayerfunktionalität in einer einzelnen Spielinstanz, danach Ausbau auf mehrere Spielinstanzen über das lokale Netzwerk (Host und Client).
        \item Reduziertes Risiko: Spiel am selben Rechner ist sehr wahrscheinlich möglich, primäres Ziel der Arbeit ist auch ohne Netzwerkfähigkeit erfüllt. Das Risiko ist selbst reduziert immer noch sehr hoch. 
        \item Tatsächliches Outcome: Risiko nicht eingetreten. Mehrere Spieler können übers Netz zusammen spielen. Das Ziel wurde nur mit beträchtlichem Aufwand erreicht (1/3 der gesamten Entwicklungszeit).
    \end{enumerate}

    \item Spiel enthält unentdeckte / unbekannte Fehler, die den Spielspass verderben (Wahrscheinlich und Kritisch)
    \begin{enumerate}
        \item Reduktionsstrategie: So viel testen wie möglich/sinnvoll, Bugs kategorisieren, sich auf kritische Bugs konzentrieren, Code Reviews.
        \item Reduziertes Risiko: Wir schätzen das reduzierte Risiko immer noch als hoch ein, da selbst wenn die Eintrittswahrscheinlichkeit stark reduziert ist; kann dieser eine unentdeckte Fehler den ganzen Spielspass vermiesen.
        \item Tatsächliches Outcome: Nach Abschluss der Alpha Testphase sind keine Fehler bekannt. Wir drücken nun die Daumen, dass Herr Hunziker keine Bugs findet.
    \end{enumerate}

\end{enumerate}

\section{Mässig}
\begin{enumerate}
    \item Elia hat noch nie mit Unity gearbeitet (Sehr wahrscheinlich und Unbedenklich)
    \begin{enumerate}
        \item Reduktionsstrategie: Elia befasst sich zu Beginn mit Unity, Marc hat Vorwissen und kann Elia unterstützen.
        \item Reduziertes Risiko: Falls Elia auf Schwierigkeiten stösst, kann dies zu zeitlichem Mehraufwand führen.
        \item Tatsächliches Outcome: Risiko nicht eingetreten. Elia konnte alles Nötige zu Beginn erlernen. Bei Problemen konnte er diese entweder durch Internetrecherchen klären oder Marc fragen. 
    \end{enumerate}
\end{enumerate}

\section{Tief}
\begin{enumerate}
    \item Wir werden selbst nicht mit unserem Spiel zufrieden sein. (Wahrscheinlich und Unbedenklich)
    \begin{enumerate}
        \item Reduktionsstrategie: Sich bewusst sein, dass wir nur zwei Schüler sind, die keine bisherigen Erfahrungen hatten. Wir dürfen uns nicht an etwas ''festfressen''.
        \item Reduziertes Risiko: Realistische Erwartungshaltung
        \item Tatsächliches Outcome: Basierend auf der uns zu verfügungsstehenden Zeit und unserem Wissen sind wir zufrieden mit unserem Produkt. Es gibt immer noch Dinge, die wir gerne verbessern und hinzufügen würden. So könnten wird noch Tausende Stunden investieren.
    \end{enumerate}
\end{enumerate}