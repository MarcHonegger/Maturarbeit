\chapter{Risiko}
Wir haben die Risiken eingeteilt in verschiedene Gefahrenstuffen.
Die Einstuffung geschah basierend auf Wahrscheinlichkeit, Wie schlimm ist der Worstcase und wie gut kann das Risiko reduziert werden.
Jedes Risiko ist dann unterteilt in:
\begin{enumerate}
    \item Risiko (Wahrscheinlichkeit und Schaden)
    \begin{enumerate}
        \item Reduktionsstrategie: Strategie um die Wahrscheinlichkeit und/oder Schaden zu reduzieren.
        \item Reduziertes Risiko: Risiko nach der Umsetzung der Reduktionsstrategie.
        \item Tatsächliches Outcome: Das tatsächliche Endresultat
    \end{enumerate}
\end{enumerate}

\section{Hoch}
\begin{enumerate}
    \item Vergange Maturaarbeiten, welche ein Videospiel als Ziel hatten, sind gescheiert (Wahrscheinlich und Kritisch)
    \begin{enumerate}
        \item Reduktionsstrategie: Sich des Risiko bewusst sein und bereit sein viel Zeit und Arbeit zu investieren
        \item Reduziertes Risiko: Das Risiko ist reduziert immernoch sehr hoch.
            Wir haben uns beide an Multiplayer fest gefressen und werden es höchst wahrscheinlich Einewegs umsetzen
        \item Tatsächliches Outcome: Wie erwartet ist die Maturaarbeit nach 250h immernoch weit entfernt von vollständig.
            Dennoch ist das Ziel erreicht von einem funktionierendem Videospiel.
    \end{enumerate}

    \item Beide Teammitglieder haben keine Ahnung von der Entwicklung einer Multiplayer-Funktion (Wahrscheinlich und Kritisch)
    \begin{enumerate}
        \item Reduktionsstrategie: Andere Spielideen erarbeiten, die ohne Multiplayer funktionieren.
        \item Reduziertes Risiko: Das Risiko ist reduziert immernoch sehr hoch.
              Wir haben uns beide an Multiplayer fest gefressen und werden es höchst wahrscheinlich Einewegs umsetzen
        \item Tatsächliches Outcome: Multiplayer ist mit Abstand der grösste Zeitfresser.
              Der Multiplayer alleine hat über 70h Arbeit, also fast ein Drittel unserer Arbeit ausgemacht.
    \end{enumerate}
\end{enumerate}

\section{Mässig}
\begin{enumerate}
    \item Elia hat noch nie mit Unity gearbeitet (Wahrscheinlich und Unbedenklich)
    \begin{enumerate}
        \item Reduktionsstrategie: Elia befasst sich zu Beginn mit Unity
        \item Reduziertes Risiko: Weiterhin kann angenommen werden, dass Elia auf gewisse Scbwierigkeiten mit Unity stossen wird.
              Jedoch sollten diese nicht viel Zeit oder Mühe benötigen. 
        \item Tatsächliches Outcome: Elia konnte alles nötige zu Begin erlernen und bei allfälligen Fragen auf das Internet oder Marc zurückgreifen.
    \end{enumerate}
\end{enumerate}

\section{Tief}
\begin{enumerate}
    \item Vergange Maturaarbeiten, welche ein Videospiel als Ziel hatten, sind gescheiert (Unwahrscheinlich und Unbedenklich)
    \begin{enumerate}
        \item Reduktionsstrategie: Sich des Risiko bewusst sein und bereit sein viel Zeit und Arbeit zu investieren
        \item Reduziertes Risiko: Das Risiko ist reduziert immernoch sehr hoch.
            Wir haben uns beide an Multiplayer fest gefressen und werden es höchst wahrscheinlich Einewegs umsetzen
        \item Tatsächliches Outcome: Wie erwartet ist die Maturaarbeit nach 250h immernoch weit entfernt von vollständig.
            Dennoch ist das Ziel erreicht von einem funktionierendem Videospiel.
    \end{enumerate}
\end{enumerate}