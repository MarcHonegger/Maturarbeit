\chapter{Risiko}
Wir haben die Risiken eingeteilt in verschiedene Gefahrenstuffen.
Die Einstuffung geschah basierend auf Wahrscheinlichkeit, Wie schlimm ist der Worstcase und wie gut kann das Risiko reduziert werden.
Jedes Risiko ist dann unterteilt in:
\begin{enumerate}
    \item Risiko (Wahrscheinlichkeit und Schaden)
    \begin{enumerate}
        \item Reduktionsstrategie: Strategie um die Wahrscheinlichkeit und/oder Schaden zu reduzieren.
        \item Reduziertes Risiko: Risiko nach der Umsetzung der Reduktionsstrategie.
        \item Tatsächliches Outcome: Das tatsächliche Endresultat
    \end{enumerate}
\end{enumerate}

\section{Hoch}
\begin{enumerate}
    \item Vergange Maturaarbeiten, welche ein Videospiel als Ziel hatten, sind gescheiert (Wahrscheinlich und Kritisch)
    \begin{enumerate}
        \item Reduktionsstrategie: Sich des Risiko bewusst sein und bereit sein viel Zeit und Arbeit zu investieren
        \item Reduziertes Risiko: Das Risiko ist reduziert immernoch sehr hoch.
            Wir haben uns beide an Multiplayer fest gefressen und werden es höchst wahrscheinlich Einewegs umsetzen
        \item Tatsächliches Outcome: Wie erwartet ist die Maturaarbeit nach 250h immernoch weit entfernt von vollständig.
            Dennoch ist das Ziel erreicht von einem funktionierendem Videospiel.
    \end{enumerate}

    \item Beide Teammitglieder haben keine Ahnung von der Entwicklung einer Multiplayer-Funktion (Wahrscheinlich und Kritisch)
    \begin{enumerate}
        \item Reduktionsstrategie: Andere Spielideen erarbeiten, die ohne Multiplayer funktionieren.
        \item Reduziertes Risiko: Das Risiko ist reduziert immernoch sehr hoch.
              Wir haben uns beide an Multiplayer fest gefressen und werden es höchst wahrscheinlich Einewegs umsetzen
        \item Tatsächliches Outcome: Multiplayer ist mit Abstand der grösste Zeitfresser.
              Der Multiplayer alleine hat über 70h Arbeit, also fast ein Drittel unserer Arbeit ausgemacht.
    \end{enumerate}
\end{enumerate}

\section{Mässig}
\begin{enumerate}
    \item Beide Teammitglieder haben keine Ahnung von der Entwicklung einer Multiplayer-Funktion (Möglich und Kritisch)
    \begin{enumerate}
    \item Reduktionsstrategie: Andere Spielideen erarbeiten, die ohne Multiplayer funktionieren.
    \item Reduziertes Risiko: Das Risiko ist reduziert immernoch sehr hoch.
          Wir haben uns beide an Multiplayer fest gefressen und werden es höchst wahrscheinlich Einewegs umsetzen
    \item Tatsächliches Outcome: Multiplayer ist mit Abstand der grösste Zeitfresser.
          Der Multiplayer alleine hat über 70h Arbeit, also fast ein Drittel unserer Arbeit ausgemacht.
    \end{enumerate}
%     \begin{enumerate}
%         \item Mitigation strategies: apply cone of uncertainty, apply definition of ready to ensure planning quality
%         \item Mitigated risk: Low
%     \end{enumerate}
% 
%     \item Poor risk management (likely and critical)  
%     \begin{enumerate}
%         \item Mitigation strategies: likelihood calculation, risk mitigation plans and monitoring of risks every planning
%         \item Mitigated risk: Low
%     \end{enumerate}
% 
%     \item Project reviewer's expectations are not aligned with project (possible and critical) 
%     \begin{enumerate}
%         \item Mitigation strategies: obtain frequent approval and acknowledgement (naturally happens for us with review meetings)
%         \item Mitigated risk: None
%     \end{enumerate}
% 
%     \item Unexpected absence of team member (unlikely and catastrophic) 
%     \begin{enumerate}
%         \item Mitigation strategies: Code changes need to be pushed on a daily basis, stories could at any point be taken over by another team member
%         \item Mitigated risk: Medium
%     \end{enumerate}
\end{enumerate}

\section{Tief}
\begin{enumerate}
    \item Reduktionsstrategie: Andere Spielideen erarbeiten, die ohne Multiplayer funktionieren.
    \item Reduziertes Risiko: Das Risiko ist reduziert immernoch sehr hoch.
          Wir haben uns beide an Multiplayer fest gefressen und werden es höchst wahrscheinlich Einewegs umsetzen
    \item Tatsächliches Outcome: Multiplayer ist mit Abstand der grösste Zeitfresser.
          Der Multiplayer alleine hat über 70h Arbeit, also fast ein Drittel unserer Arbeit ausgemacht.
%     \begin{enumerate}
%         \item Mitigation strategies: code reviews, clear coding standards, apply definition of done
%         \item Mitigated risk: None
%     \end{enumerate}
% 
%     \item Lack of ownership (possible and marginal) 
%     \begin{enumerate}
%         \item Mitigation strategies: setting clear responsibilities for roles
%         \item Mitigated risk: None
%     \end{enumerate}
% 
%     \item Losing sight of documentation tasks (possible and marginal) 
%     \begin{enumerate}
%         \item Mitigation strategies: documentation strategy, documentation part of definition of done
%         \item Mitigated risk: Low
%     \end{enumerate}
% 
%     \item Failure of hardware like personal devices, OST GitLab, Jira, hosted environment (rare, catastrophic) 
%     \begin{enumerate}
%         \item Mitigation strategies: Code changes need to be pushed on a daily basis
%         \item Mitigated risk: Low
%     \end{enumerate}
\end{enumerate}