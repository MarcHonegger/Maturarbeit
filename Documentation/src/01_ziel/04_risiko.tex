\chapter{Risiko}
Wir haben die Risiken eingeteilt in verschiedene Gefahrenstufen.
Die Einstufung geschah basierend auf Wahrscheinlichkeit, wie schlimm ist der Worstcase und wie gut kann das Risiko reduziert werden.
Jedes Risiko ist dann unterteilt in:
\begin{enumerate}
    \item Risiko (Wahrscheinlichkeit und Schaden)
    \begin{enumerate}
        \item Reduktionsstrategie: Strategie um die Wahrscheinlichkeit und/oder Schaden zu reduzieren.
        \item Reduziertes Risiko: Risiko nach der Umsetzung der Reduktionsstrategie.
        \item Tatsächliches Outcome: Das tatsächliche Endresultat
    \end{enumerate}
\end{enumerate}

\section{Hoch}
\begin{enumerate}
    \item Vergange Maturitätsarbeiten, welche ein Videospiel als Ziel hatten, sind gescheitert (Wahrscheinlich und Kritisch)
    \begin{enumerate}
        \item Reduktionsstrategie: Sich des Risikos bewusst sein und bereit sein, viel Zeit und Arbeit zu investieren
        \item Reduziertes Risiko: Das Risiko ist reduziert immer noch sehr hoch.
            Wir haben uns beide an Multiplayer fest gefressen und werden es höchst wahrscheinlich trotzdem umsetzen
        \item Tatsächliches Outcome: Wie erwartet ist die Maturaarbeit nach 250h immer noch weit entfernt von vollständig.
            Dennoch ist das Ziel erreicht von einem funktionierendem Videospiel.
    \end{enumerate}

    \item Beide Teammitglieder haben keine Ahnung von der Entwicklung einer Multiplayer-Funktion (Wahrscheinlich und Kritisch)
    \begin{enumerate}
        \item Reduktionsstrategie: Andere Spielideen erarbeiten, die ohne Multiplayer funktionieren.
        \item Reduziertes Risiko: Das Risiko ist reduziert immer noch sehr hoch.
              Wir haben uns beide an Multiplayer fest gefressen und werden es höchst wahrscheinlich trotzdem umsetzen
        \item Tatsächliches Outcome: Multiplayer ist mit Abstand der grösste Zeitfresser.
              Der Multiplayer alleine hat über 100h Arbeit, also fast ein Drittel unserer Arbeit ausgemacht.
    \end{enumerate}

    \item Es könnten sich Bugs in der abgabebereiten Version verstecken (Wahrscheinlich und Kritisch)
    \begin{enumerate}
        \item Reduktionsstrategie: So viel testen wie nur möglich. Sich ausserdem auf die kritischen Bugs konzentrieren. 
        \item Reduziertes Risiko: Das Risiko ist selbst reduziert sehr hoch. Wir wollen ein perfektes Game abgeben.
        \item Tatsächliches Outcome: Wir haben unser Möglichstes gegeben. Wir können nur die Daumen drücken, dass Herr Hunziker keine neuen Bugs findet.
    \end{enumerate}

    \item Die Version für die Abgabe könnte aus unbekannten Gründen crashen (Unwahrscheinlich und Kritisch)
    \begin{enumerate}
        \item Reduktionsstrategie: So viel testen wie nur möglich.
        \item Reduziertes Risiko: Das Risiko ist selbst reduziert sehr hoch. Man konnte bereits des Öfteren beobachten, wie sich eine schlechte erste Version auf eine Firma auswirken kann.
        \item Tatsächliches Outcome: Wir haben unser Möglichstes gegeben. Wir könnten nur noch die Daumen drücken und hoffen, dass es keine fatalen Fehler gibt.
    \end{enumerate}

\end{enumerate}

\section{Mässig}
\begin{enumerate}
    \item Elia hat noch nie mit Unity gearbeitet (Wahrscheinlich und Unbedenklich)
    \begin{enumerate}
        \item Reduktionsstrategie: Elia befasst sich zu Beginn mit Unity
        \item Reduziertes Risiko: Weiterhin kann angenommen werden, dass Elia auf gewisse Schwierigkeiten mit Unity stossen wird.
              Jedoch sollten diese nicht viel Zeit oder Mühe benötigen. 
        \item Tatsächliches Outcome: Elia konnte alles Nötige zu Beginn erlernen und bei allfälligen Fragen auf das Internet oder Marc zurückgreifen.
    \end{enumerate}
\end{enumerate}

\section{Tief}
\begin{enumerate}
    \item Wir werden selbst nicht mit unserem Spiel zufrieden sein. (Wahrscheinlich und Unbedenklich)
    \begin{enumerate}
        \item Reduktionsstrategie: Sich bewusst sein, dass wir nur 2 Schüler sind, die keine bisherigen Erfahrungen hatten. Wir dürfen uns nicht an etwas ''festfressen''.
        \item Reduziertes Risiko: Es stellt sich das Risiko, dass wir unsere Zeit verschwenden. Wir müssen teilweise einfach weitermachen.
        \item Tatsächliches Outcome: Man ist nie richtig fertig mit einem Spiel. Es gibt immer Dinge, die man Verbessern könnte. So könnten auch wir noch tausende weitere Stunden investieren.
    \end{enumerate}
\end{enumerate}