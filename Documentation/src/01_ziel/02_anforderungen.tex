\chapter{Anforderungen}

\section{Anforderungen an die Maturitätsarbeit}
% \includegraphics[height=18cm]{resources/diagrams/use-case}
\href{https://www.kzo.ch/fileadmin/internet/pdf_internet/Unterricht/Maturjahr/Reglement_Maturitaetsarbeit_2023.pdf}{Die Anforderungen der Schule} halten sich in Grenzen. Es gibt nur wenige genauen Anforderungen
und die, die es gibt, sind relativ vage formuliert. Die drei Hauptanforderungen der KZO sind:
\begin{enumerate}
    \item \textbf{Zeit} \\
        \textit{Ungefähr eine Lektion pro Woche soll während den Semestern 5.2 und 6.1. als Zeitaufwand veranschlagt werden}\\Diese Stunde wird bei den Wochenlektionen dieser Semester angerechnet.
        Jedoch ist diese Anforderung für uns nicht schwer zu erfüllen, denn unser Projekt wird eine sehr umfangreiche und arbeitsintensive Maturitätsarbeit sein.
        Wir werden diese Zeitanforderung ohne Probleme erreichen und bei Weitem überschreiten.
    \item \textbf{Begleitperson} \\
        \textit{Jede Maturitätsarbeit benötigt eine Begleitperson. }\\Diese Lehrperson bewertet die Arbeit und unterstützt die Schüler bei Bedarf.
        Wir sind bereits früh auf Herrn Albert Kern (Informatiklehrer der KZO) zugegangen.
        Er hat uns an Herrn Martin Hunziker (Leiter der IT der KZO) verwiesen. Dieser nahm uns sehr gerne mit der Anmerkung auf, dass die Umsetzung unserer Idee sehr schwierig sein werde.
        %Letzteres, also Unterstützung, war bei uns sehr schwierig, denn keine Lehrperson kennt sich mit Unity und C\# genug gut aus.
    \item \textbf{Plagiat} \\
        \textit{Arbeiten dürfen nicht einfach kopiert werden. }\\Wir haben uns vereinzelt bei gewissen Problemen online inspirieren lassen. Unsere verwendeten Lösungen sind jedoch keine Plagiate.
        Unser gesamter Code sowie unsere schriftliche Arbeit sind von uns selbst geschrieben.
\end{enumerate}

\section{Anforderungen an das Spiel}
\label{chap:foo}
Gewisse \glspl{Feature} sind für das Funktionieren des Spiels wichtiger als andere.
Zum Beispiel sind für uns Einstellungen, Truppen und \gls{Multiplayer} notwendiger als die Monetarisierung.
Deshalb ist der folgende Abschnitt in vier Stufen unterteilt:
\begin{enumerate}
    \item \textbf{Notwendig:}
        Ohne diese Features läuft das Spiel nicht oder ist sehr mangelhaft,
        voller Fehler und unspielbar. Ohne diese Funktionen ist die Idee des Spieles nicht erkennbar.
    \item \textbf{Anvisiert:}
        Diese Features sind unser Ziel. Sie wurden Herrn Hunziker angekündigt und sind alle in unserem Plan festgehalten.
        Diese Funktionen sollen dazu führen, dass das Spiel
        gut spielbar ist und es Spass macht. Sie sind auch dafür essenziell, das Spiel nicht zu monoton zu gestalten und sollten
        alle bis zum Abgabetermin der schriftlichen Arbeit vollendet sein.
    \item \textbf{Möglich:}
        Dies sind Möglichkeiten, den Spielspass auf ein neues Niveau zu heben. Allerdings ist der Zeitaufwand für die Programmierung dieser Funktionen sehr gross.
        Keine dieser Anforderungen ist für das Spielen nötig,
        kann das Spielerlebnis jedoch spannender und ausgereifter machen. Diese
        Ideen sind teilweise nur Konzepte, die noch bis zur mündlichen Präsentation zu erreichen sind.
        Vereinzelt können diese, wie die anvisierten Anforderungen, noch bis zur schriftlichen Abgabe erreicht werden.
    \item \textbf{Verworfen:}
        Weitere Ideen, die wir hatten, wir jedoch als unmöglich oder nicht sinnvoll erachten.
\end{enumerate}

\subsection{Notwendig}
\begin{itemize}
    \item \textbf{Systemanforderungen} \\
        Unser Spiel soll auf den meisten modernen Computern funktionieren. Es soll auf macOS und
        Windows fliessend gespielt werden können (mindestens 60\gls{FPS}).
    \item \textbf{Benutzeroberfläche} \\
        Die Startseite soll Folgendes enthalten: einen ''Spielen''-Knopf, einen ''Einstellungen''-Knopf, einen
        ''Credits''-Knopf und einen ''Verlassen''-Knopf. Die Benutzeroberfläche soll intuitiv bedienbar sein und
        schön aussehen. Sie soll mit der Maus bedienbar sein. Andere Eingabemöglichkeiten
        werden nicht unterstützt.
    \item \textbf{\gls{Lokal}er Multiplayer} \\
        Das Spiel soll im lokalen Netz gespielt werden können, so zum Beispiel mit Freunden oder Familie.
        Es sollte mithilfe der IP-Adresse innerhalb des gleichen Netzes zusammengespielt werden können.
    \item \textbf{Einstellungen} \\
        Die Auflösung, Vollbild und Lautstärke müssen im UI eingestellt werden können.
    \item \textbf{\gls{Kamera}} \\
        Die Kamera ist beweglich und das ganze Schlachtfeld ist sichtbar.
    \item \textbf{Truppen} \\
        Es braucht Truppen, die für eine*n Spieler*in kämpfen können. Diese sind für den Spielverlauf notwendig.
        Und ohne sie hat das Spiel keinen Sinn.
    \item \textbf{Gewinnmöglichkeit} \\
        Es soll eine Möglichkeit geben, das Spiel zu gewinnen oder zu verlieren und es somit zu beenden. Diese Anforderung kann sehr simpel
        mit einer Linie verwirklicht werden, welche beim Überschreiten einer Truppe das Spiel beendet.
    \item \textbf{\glspl{Lane}} \\
        Der*die Spieler*in soll auswählen dürfen, an welcher Stelle er seine Truppe heraufbeschwören will. Dabei soll er*sie die Auswahl zwischen vier Linien haben. Diese Linien sind in die Tiefe verschobenen Abgrenzungen.
    \item \textbf{Synchronisation} \\
        Unser Spiel soll in Echtzeit spielbar sein, deswegen müssen die beiden Spieler*innen synchronisiert dasselbe sehen. Eine Desynchronisation wäre verheerend. 
    \item \textbf{\gls{Crossplay}}
        Wir wollen die Möglichkeit erreichen, einen Windows Rechner und einem macOS Rechner zu verbinden und zusammenzuspielen.
\end{itemize}

\subsection{Anvisiert}
\begin{itemize}
    \item \textbf{\gls{Helden}} \\
        Beide Spieler*innen haben eine Art Truppe, welche ihre Siegeslinie beschützt. Sie haben einen Racheeffekt, welcher die
        Lane ausradiert, damit das Spiel nicht sofort vorbei ist.
    \item \textbf{Design} \\
        Das Spiel sollte vom Aussehen her ''was hergeben''. Es sollte nicht wie ein Prototyp, sondern wie ein 
        vollendetes Spiel aussehen. 
    \item \textbf{Deck} \\
        Zu Beginn können die Spieler*innen ihr eigenes Deck aus einer Auswahl von Karten zusammenstellen. Im Spiel werden aus diesem Kartendeck per Zufall Karten gezogen.
    \item \textbf{Bot} \\
        Dies ist ein sehr simpler Algorithmus, um alleine gegen den Computer zu spielen. Die Schwierigkeitsstufe kann zwischen \textit{'Einfach', 'Mittel', 'Schwierig'} gewählt werden.
        Angepasst an den Schwierigkeitsgrad werden zufällig Truppen geschickt.
    \item \textbf{Truppen}\\
    Es gibt verschiedene Truppenarten:
    \begin{itemize}
        \item \textbf{Nahkampftruppe:}
            Wenig Reichweite
        \item \textbf{Fernkampftruppe:}
            Greift auf grosse Reichweite an. Sie schiesst Projektile (z.B. ein Bogenschütze, der mit Pfeilen schiesst).
        \item \textbf{Suizidtruppe:}
            Stirbt bei Berührung mit einem Gegner und löst einen Effekt aus.
    \end{itemize}
    \item \textbf{Effekte}
    \begin{itemize}
        \item \textbf{Gift:}
            Zeitlich limitierter und wiederholender Schadenseffekt. \\Eine visuelle Markierung soll vorhanden sein
            und der Gifteffekt darf sich nicht kumulieren.
        \item \textbf{Dornen:}
            Truppen, welche Charaktere mit Dornen attackieren, erhalten Schaden. Der Effekt soll nach Auswahl auf Nahkämpfer,
            Fernkämpfer oder beides wirken.
        \item \textbf{Rache:}
            Truppen mit dem Effekt Rache lösen einen Effekt nach ihrem Tod aus. Beispielsweise kann durch diesen Tod eine Truppe beschworen oder Schaden verursacht werden.
    \end{itemize}
\end{itemize}

\subsection{Möglich}
\begin{itemize}
    \item \textbf{Zauber} \\
        Alternativ zum Herbeirufen einer Truppe können gewisse Karten einen Effekt auslösen (Zauber).\\
        Beispiele für Zauber:
            \begin{itemize}
                \item ''Heile deine Helden um 5 Lebenspunkte''
                \item ''Gib deinen Helden 5 Rüstungspunkte''
                \item ''Verursache allen gegnerischen Truppen 5 Schadenspunkte''
                \item ''Ziehe 3 Karten''
            \end{itemize}
    \item \textbf{Helden} \\
        Eine Auswahl von Helden, mit unterschiedlichem Schaden. \\Leben und Effekte des jeweiligen Helden würden das Spiel nochmals spannender machen. Auch sollten gewisse Truppen auf bestimmte Helden limitiert sein. 
    \item \textbf{Truppen}\\
        Weitere Truppen können jederzeit erstellt werden. Hier ist kein Limit gesetzt. Leben, Schaden, Darstellung, Effekt und vieles mehr kann angepasst werden.
    \item \textbf{Effekte}
    \begin{itemize}
        \item \textbf{Wiederbelebung:}
            Die Truppe wird sofort an Ort und Stelle oder mit einer Verzögerung am Startpunkt wiederbelebt.
        \item \textbf{Rüstung:}
            Die Truppe hat zusätzliche Rüstungspunkte zu den Lebenspunkten. Die Rüstung wird zuerst abgezogen und hat im Gegensatz zum Leben keine Limite.
        \item \textbf{Aura:}
            Die Truppe fügt gegnerischen Truppen in einem gewissen Radius permanent Schaden zu.
    \end{itemize}
    \item \textbf{Design} \\
        Wenn ausreichend Zeit vorhanden ist, kann das Design immer weiter verbessert werden. Hier geht es um den Feinschliff, z.B. mehr Hintergründe, Details und mehr selbsterstellte Designs.
    \item \textbf{\gls{Tutorial}} \\
        Eine Anleitung für Anfänger*innen wäre sehr schön. Sie soll neuen Spielern*innen das Anfangen erleichtern. Sie sollte aber auch überspringbar sein,
        damit erfahrene Spieler*innen das Tutorial nicht erneut spielen müssen. Es soll weder allzu lang noch zu schwierig sein. Idealerweise soll es ''anhand von Gameplay'' alle Mechaniken des Spiels
        erklären.
    \item \textbf{Mehrsprachig} \\
        Das Spiel soll in Englisch, Deutsch und Französisch verfügbar sein.
    \item \textbf{\gls{Ingame}-Chatfenster}\\
        Ein Textfeld, in dem man mit dem/der Gegner*in chatten kann, würde einen zusätzlichen Spassfaktor bieten.
    \item \textbf{Monetarisierung} \\
    Es sind uns drei Möglichkeiten eingefallen, wie wir mit dem Spiel Geld verdienen könnten. Es folgen jeweils die Vor- und Nachteile:
    \begin{enumerate}
        \item \textbf{Kaufbare Gegenstände}
        \begin{itemize}
            \item[+] Vielseitige und nachhaltige Monetarisierung. Neue Features bedeuten auch neue Einnahmemöglichkeiten. Die käuflichen Inhalte sollten
                        auch erspielbar sein. Somit kann bezahlen zwar zu Vorteilen führen, diese können jedoch mit einer hohen Spielzeit trotzdem erreichbar sein.
            \item[-] \gls{Pay-to-Win}
        \end{itemize}
        \item \textbf{\gls{Skins}}
        \begin{itemize}
            \item[+] Weit verbreitet und sehr beliebt bei Spieler*innen geben sie keinem*keiner Spieler*in Vorteile.
            \item[-] Wenig Nutzen und sehr aufwendig. Neue Designs zu erstellen, wäre bei uns nicht sinnvoll.
                        Wir möchten noch sehr viele Features implementieren und benötigen dafür viele zeitlichen Ressourcen. Zudem sind wir keine spezialisierten Designer. Der Aufwand für uns, eine ansprechende und brauchbare Darstellung zu erschaffen, steht in keinem sinnvollen Verhältnis zum Mehrwert für das Spiel.
        \end{itemize}
        \item \textbf{Kostenpflichtiges Spiel}
        \begin{itemize}
            \item[+] Einmalige Bezahlung bei Spielkauf, was für viele Nutzer*innen lukrativer ist. Die gilt jedoch vor allem bei \gls{Singleplayer}-Spielen.
            \item[-] Wir nehmen an, dass wenige Nutzer*innen bereit wären, Geld für unser Spiel zu bezahlen.
                    Dafür wird es vorerst nicht genug ausgereift sein.
                    Auch ist diese Methodik nicht nachhaltig und führt nur zu einer einmaligen Geldspritze.
                    Viele grössere Spiele führen deshalb später
                    \glspl{DLC} ein, um das Spiel zu erweitern.
                    Jedoch ist dies bei einem Multiplayer Spiel Pay-to-Win.
                    Wir müssen mit hoher Wahrscheinlichkeit vorerst Geld investieren, um unser Spiel anbieten zu können.
        \end{itemize}
    \end{enumerate}
\end{itemize}

\subsection{Bereits früh verworfen}
\begin{itemize}
    \item \textbf{\gls{Online} Multiplayer} \\
        Besser als ein Multiplayer, der limitiert auf dasselbe Netz ist, ist ein Multiplayer, der weltweit verfügbar ist. Jedoch ist die direkte Verbindung zwischen zwei Computern, die sich in unterschiedlichen Netzen befinden, dank der heutigen \glspl{Firewall} von Routern und Computern für uns nahezu unmöglich. 
        Es würden sich viele weitere Probleme ergeben, wie zum Beispiel Port-Forwarding. Dies ist einer der Gründe, weshalb ein Server sehr praktisch wäre. Mit diesem wäre dieses
        Problem gelöst. Server sind aber teuer und aufwendig zu programmieren und zu warten. 
    \item \textbf{\gls{Shop}} \\
        Nicht alle Karten sind von Beginn an freigegeben, sondern müssen freigespielt werden.
        So kann zum Beispiel eine Ingame Währung erspielt werden und damit im Shop Karten oder sonstige Dinge gekauft werden.
        Der Shop kann mit Zahlungsmethoden ausgestattet werden und so zur Monetarisierung beitragen.
    \item \textbf{Kampagne} \\
        Im Singleplayer spielt man gegen bestimmte vorprogrammierte Gegner. Sie sollen immer stärker werden und bestimmte Herausforderungen mit sich bringen.
        Bei einem Sieg werden Belohnungen verteilt.
    \item \textbf{\gls{Anti-Cheat}} \\
        In allen grösseren Spielen ist eine sehr komplexe Software vorhandeln, um das Schummeln zu verhindern.
        Teilweise geschieht dies auf Systemebene oder sogar auf Kernel-Level.
    \item \textbf{Errungenschaften} \\
        Durch Erfüllung von Aufgaben erhält man Belohnungen (z.B. Freischalten von bestimmten Karten). 
        Die möglichen und bereits erreichten Errungenschaften sollen in einer Übersicht dargestellt werden. 
        Diese Erweiterung ist keinesfalls notwendig und ein absolutes nice-to-have Feature.
\end{itemize} 

% \includegraphics[height=10cm, width=\textwidth]{resources/mockups/mockup-login}