\chapter{Anforderungen}

\section{KZO}
% \includegraphics[height=18cm]{resources/diagrams/use-case}
Die Anforderungen der Schule halten sich in Grenzen. Zum einen gibt es fast keine
genauen Anforderungen und die Anforderungen, die es gibt, sind relativ tief gehalten. Die drei Anforderungen der KZO sind:
\begin{enumerate}
    \item \textbf{Zeit} \\
        Geplant ist ungefähr eine Lektion pro Woche, denn diese wir dem Stundenplan für das Semester abgezogen.
        Jedoch ist diese Anforderung bei uns nicht relevant, denn unser Projekt ist eine riesige Maturitätsarbeit.
        Wir werden ohne Probleme diese Zeitanforderung erreichen und auch bei Weitem überschreiten.
    \item \textbf{Begleitperson} \\
        Jede Maturitätsarbeit braucht eine Begleitperson. Also eine Lehrperson, welche die Arbeit bewertet und einen unterstützt.
        Letzteres, also Unterstützung, war bei uns sehr schwierig, denn keine Lehrperson kennt sich mit Unity und C\# genug gut aus.
        Wir sind bereits früh auf Herrn Kern, ein Informatiklehrer, zugegangen.
        Er hat uns an Martin Hunziker, Leiter der IT, weitergeleitet. Dieser nahm uns sehr gerne auf, mit der Anmerkung, dass unsere Arbeit sehr schwierig sein wird, umzusetzen.
    \item \textbf{Plagiat} \\
        Arbeiten dürfen nicht einfach kopiert werden. Wir haben uns zwar bei gewissen Problemen online inspirieren lassen, jedoch sind diese keine Plagiate.
        Der grösste Teil unseres Codes und alles unserer schriftlichen Arbeit sind von uns geschrieben.
\end{enumerate}

\section {Unsere eigenen Anforderungen}
Gewisse Features sind für das Funktionieren des Spiels wichtiger als andere.
Zum Beispiel sind Einstellungen, Truppen und Multiplayer notwendiger als die Monetarisierung, zumindest nach unserer Ansicht.
Deshalb ist dieser Abschnitt in 4 Gruppen unterteilt:
\begin{enumerate}
    \item \textbf{Notwendig:}
        Ohne diese Features läuft das Spiel nicht oder ist sehr mangelhaft,
        voller Fehler und unspielbar. Ohne diese Features ist Idee des Spieles nicht erkennbar.
    \item \textbf{Anvisiert:}
        Diese Features sind unser Ziel. Sie wurden Herr Hunziker, also der Begleitperson, angekündigt und sind alle in einem Plan festgehalten.
        Diese Ziele sollen dazu führen, dass das Spiel
        gut spielbar ist und es Spass macht. Sie sind auch dafür essenziell, das Spiel nicht zu monoton zu gestalten und sollten
        alle bis zum Abgabetermin der schriftlichen Arbeit vollendet sein.
    \item \textbf{Möglichkeiten:}
        Möglichkeiten das Spiel noch auf die Spitze zu bringen. Keine dieser Anforderungen ist nötig
        für das Spielen, jedoch können sie das Spielerlebnis spannender und ausgereifter machen. Diese
        Ziele sind entweder in den Sternen geschrieben oder zu erreichen bis zur mündlichen Präsentation.
        Vereinzelt können diese, wie die anvisierten Anforderungen, noch bis zur schriftlichen Abgabe erreicht werden.
    \item \textbf{Verworfen:}
        Weitere Ideen, welche wir hatten, jedoch als unmöglich oder nicht sinnvoll erachten.
\end{enumerate}

\subsection*{Notwendig}
\begin{itemize}
    \item \textbf{System} \\
        Unser Spiel sollte auf den meisten modernen Geräten funktionieren. Es soll auf macOS und
        Windows fliessend gespielt werden können. Also mit mindestens 60FPS.
    \item \textbf{Benutzeroberfläche} \\
        Eine Startseite mit einem ''Spielen''-Knopf, einem ''Einstellungen''-Knopf, einem
        ''Credits''-Knopf und einem ''Verlassen''-Knopf. Sie soll intuitiv sein und einigermassen
        schön aussehen. Die Benutzeroberfläche soll mit Touch bedienbar sein, andere Eingabemöglichkeiten
        werden nicht unterstützt.
    \item \textbf{Lokaler Multiplayer} \\
        Das Spiel soll im lokalen Netz gespielt werden können. So zum Beispiel mit Freunden oder Familie.
        Es sollte mithilfe der IP-Adresse innerhalb des gleichen Netzes zusammengespielt werden können.
    \item \textbf{Einstellungen} \\
        Auflösung, Vollbild und Lautstärke müssen eingestellt werden können.
    \item \textbf{Kamera} \\
        Die Kamera ist beweglich und das ganze Schlachtfeld ist sichtbar.
    \item \textbf{Truppen} \\
        Es braucht Truppen, die für einen kämpfen können. Diese sind notwendig für den Verlauf des Spieles
        und ohne sie hat das Spiel keinen Sinn.
    \item \textbf{Gewinnmöglichkeit} \\
        Eine Chance das Spiel zu gewinnen oder verlieren und es somit zu beenden. Dies kann sehr simpel
        mit einer Linie vollendet werden, welche beim Überschreiten das Spiel beendet.
    \item \textbf{Lanes} \\
        Unser Spiel ist zwar 2.5D, aber hat dennoch eine Tiefe.
    \item \textbf{Synchronisation} \\
        Unser Spiel soll in Echtzeit spielbar sein, deswegen müssen die beiden Spieler Synchronisiert das Gleiche sehen. Eine De-Synchronisation wäre verheerend. 
    \item \textbf{Crossplay}
        Die Möglichkeit sich mit einem Windows Rechner und einem macOS Rechner zu verbinden und zusammenzuspielen.
\end{itemize}

\subsection*{Anvisiert}
\begin{itemize}
    \item \textbf{Helden} \\
        Beide Spieler haben eine Art Truppe, welche ihre Siegeslinie beschützt. Sie haben einen Racheeffekt, welcher die
        Lane ausradiert, damit das Spiel nicht sofort vorbei ist.
    \item \textbf{Design} \\
        Das Spiel sollte vom Aussehen her was hergeben. Es sollte nicht wie ein Prototyp, sondern wie ein 
        vollendetes Spiel aussehen. 
    \item \textbf{Deck} \\
        Zu Beginn Spieler könne ihr eigenes Deck aus einer Auswahl von Karten zusammenstellen. Im Spiel werden davon per Zufall Karten gezogen.
    \item \textbf{Bot} \\
        Ein sehr simpler Algorithmus um alleine gegen den Computer zu spielen auf den Schwierigkeitsstufen \textit{'Einfach', 'Mittel', 'Schwierig'}.
        Es sollen rein zufällig Truppen geschickt werden, bei höherer Schwierigkeit mehr.
    \item \textbf{Truppen}
    \begin{itemize}
        \item \textbf{Nahkampf}
            Eine Truppe mit wenig Reichweite.
        \item \textbf{Fernkampf}
            Eine Truppe die auf Reichweite angreift. Sie scheisst Projektive, zum Beispiel ein Bogenschütze,
            der mit Pfeilen schiesst.
        \item \textbf{Suizid}
            Eine Truppe die bei Berührung mit einem Gegner stirbt und einen Effekt auslöst.
    \end{itemize}
    \item \textbf{Effekte}
    \begin{itemize}
        \item \textbf{Gift:}
            Zeitlich limitierter und wiederholender Schadenseffekt. Eine visuelle Markierung soll vorhanden sein
            und das gleiche Gift, also der gleichen Truppe, soll nur einmal auf jemanden sein.
        \item \textbf{Dornen:}
            Truppen, welche Charaktere mit Dornen attackieren, erhalten schaden. Soll nach Auswahl auf Nahkämpfer,
            Fernkämpfer oder beides wirken.
        \item \textbf{Rache:}
            Truppen mit Rache haben einen Effekt nach ihrem Tod. Zum Beispiel eine Truppe beschwören oder Schaden verursachen.
    \end{itemize}
\end{itemize}

\subsection*{Möglichkeiten}
\begin{itemize}
    \item \textbf{Zauber} \\
        Als Alternative sollen nicht alle Karten einfach eine Truppe herbeirufen. Gewisse Karten sollten nur einen Effekt auslösen.
        Beispiele für Zauber:
            "Heile deine Helden um 5 leben"
            "Gib deinen Helden 5 Rüstung"
            "Verursache allen gegnerischen Truppen 5 Schaden"
            "Ziehe 3 Karten"
    \item \textbf{Monetarisierung} \\
    Hierfür sind uns drei Möglichkeiten eingefallen, hier jeweils die Vor- und Nachteile:
    \begin{enumerate}
        \item \textbf{Kaufbare Gegenstände}
        \begin{itemize}
            \item[+] Vielseitige und Nachhaltige Monetarisierung. Neue Features, bedeuten auch neue Geldmöglichkeit. Die Inhalte sollten
                     auch erspielbar sein, somit kann bezahlen zwar zu Vorteilen führen, jedoch mit viel Spielen trotzdem erreichbar sein.
            \item[-] Pay-to-Win
        \end{itemize}
        \item \textbf{Skins}
        \begin{itemize}
            \item[+] Weit verbreitet und sehr beliebt bei Spielern. Gibt keinem Spieler Vorteile
            \item[-] Wenig Nutzen und sehr aufwändig. Neue Designs zu erstellen, wäre bei uns nicht so sinnvoll.
                     Wir haben noch sehr viel Features zu implementieren und sind nicht Designer. Für uns braucht es sehr
                     viel Zeit eine brauchbare Darstellung zu erstellen und der Mehrwert, welcher dem Spiel beigetragen wird, ist marginal.
        \end{itemize}
        \item \textbf{Kostenpflichtiges Spiel}
        \begin{itemize}
            \item[+] Einmalige Bezahlung, was für viele Nutzer lukrativer ist. Jedoch gilt dies vor allem bei Singleplayer Spielen.
            \item[-] Wir nehmen an, dass wenige Leute bereit wären, Geld für unser Spiel zu bezahlen.
                    Dafür wird es nicht genug ausgereift sein.
                    Auch ist diese Methodik nicht nachhaltig und führt nur zu einer einmaligen Geldspritze.
                    Viele grössere Spiele führen deshalb später
                    DLCs ein, um das Spiel zu erweitern.
                    Jedoch ist dies bei einem Multiplayer Spiel Pay-to-Win.
                    Abschliessen müssten wir höchstwahrscheinlich selbst Geld vorauswerfen, um unser Spiel anbieten zu können, z.B. auf Steam.
        \end{itemize}
        
    \end{enumerate}
    \item \textbf{Helden} \\
        Eine Auswahl von Helden, mit unterschiedlichem Schaden, Leben und Effekten, würde das Spiel nochmals spannender machen. Auch sollten gewisse Truppen auf bestimmte Helden limitiert sein. 
    \item \textbf{Design} \\
        Wenn die Zeit vorhanden ist, kann das Design immer verbessert werden. Hier gehts es aber um den Feinschliff, z.B. mehr Hintergründe, und mehr Dinge selbst zu designen.
    \item \textbf{Tutorial} \\
        Eine Anleitung für Anfänger wäre sehr schön. Sie soll neuen Spielern das Anfangen erleichtern. Es sollte aber auch überspringbar sein,
        damit alte Spieler, es nicht nochmals spielen müssen. Es soll nicht lange sein und nicht schwierig. Dennoch soll es alle Mechaniken des Spiels
        erklären, am besten Anhand von Gameplay.
    \item \textbf{Truppen}
        Weitere Truppen können jederzeit erstellt werden. Hier ist kein Limit gesetzt. Leben, Schaden, Darstellung, Effekt und vieles mehr kann angepasst werden.
    \item Effekte
    \begin{itemize}
        \item \textbf{Wiederbelebung:}
            Die Truppe wird wiederbelebt, sofort an Ort und Stelle oder mit einer Verzögerung am Startpunkt.
        \item \textbf{Rüstung:}
            Die Truppe hat zusätzlichen zu den Leben auch Rüstung. Die Rüstung wird zuerst abgezogen und hat im Gegensatz zum Leben keine Limite.
        \item \textbf{Aura:}
            Die Truppe fügt gegnerischen Truppen in einem gewissen Radius permanent Schaden zu.
    \end{itemize}
    \item \textbf{Mehrsprachig} \\
        Das Spiel soll in Englisch, Deutsch und Französisch spielbar sein.
    \item \textbf{Ingame-Chatfenster}\\
        Ein Textfeld, in dem man sich mit dem Gegner unterhalten kann.
\end{itemize}

\subsection*{Bereits früh verworfen}
\begin{itemize}
    \item \textbf{Online Multiplayer} \\
        Besser als ein Multiplayer, der limitiert auf dasselbe Netz ist, ist ein Multiplayer, der weltweit verfügbar ist. Jedoch ist von einem Computer auf einen anderen zu
        verbinden dank der Firewall von Router und Computer, nahezu unmöglich. Es würden sich viele weitere Probleme ergeben, wie zum Beispiel Port-Forwarding. Dies ist einer der Gründe, weshalb ein Server sehr praktisch ist. Mit diesem wäre dieses
        Problem gelöst. Server sind aber teuer und aufwändig. 
    \item \textbf{Shop} \\
        Nicht alle Karten sind von Beginn an freigeben, sondern müssen freigespielt werden.
        So kann zum Beispiel eine Ingame Währung erspielt werden und damit im Shop Karten oder sonstige Dinge gekauft werden.
        Der Shop kann mit Zahlungsmethoden ausgestattet werden und so zur eventuellen Monetarisierung beitragen.
    \item \textbf{Kampagne} \\
        Singleplayer gegen bestimmte vorprogrammierte Gegner. Sie sollen immer schwieriger werden und bestimmte Herausforderungen mit sich bringen.
        Bei Vollendung werden Belohnungen verteilt.
    \item \textbf{Anti-Cheat} \\
        Eine sehr komplexe Software um das Schummeln zu verhindern. Ist in allen grösseren Spielen vorhanden. Teilweise sogar auf Systemebene, oder sogar auf Kernel-Level,
        gespeichert und ausgeführt.
    \item \textbf{Errungenschaften} \\
        Eine Übersicht und die Möglichkeit die Errungenschaften zu erfüllen.
        Gegebenenfalls Belohnungen verteilen, wie zum Beispiel bestimmte Karten freischalten.
        Diese Erweiterung ist keines Falls notwendig und ein absolutes nice-to-have Feature.
\end{itemize}

% \includegraphics[height=10cm, width=\textwidth]{resources/mockups/mockup-login}