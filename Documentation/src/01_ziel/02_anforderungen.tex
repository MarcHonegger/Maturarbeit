\chapter{Anforderungen}

\section{Anforderungen an die Maturitätsarbeit}
% \includegraphics[height=18cm]{resources/diagrams/use-case}
Die Anforderungen der Schule (\url{https://www.kzo.ch/fileadmin/internet/pdf_internet/Unterricht/Maturjahr/Reglement_Maturitaetsarbeit_2023.pdf})halten sich in Grenzen. Einerseits gibt es fast keine
genauen Anforderungen und die, die es gibt, sind relativ wage formuliert. Die drei Hauptanforderungen der KZO sind:
\begin{enumerate}
    \item \textbf{Zeit} \\
        Geplant ist ungefähr eine Lektion pro Woche während den Semestern 5.2 und 6.1. \\Diese Stunde wird bei den Wochenlektionen dieser Semester angerechnet.
        Jedoch ist diese Anforderung für uns nicht relevant, denn unser Projekt wird eine riesige Maturitätsarbeit.
        Wir werden diese Zeitanforderung ohne Probleme erreichen und bei Weitem überschreiten.
    \item \textbf{Begleitperson} \\
        Jede Maturitätsarbeit benötigt eine Begleitperson. \\Diese Lehrperson bewertet die Arbeit und unterstützt die Schüler bei Bedarf.
        Wir sind bereits früh auf Herrn Kern(Informatiklehrer KZO) zugegangen.
        Er hat uns an Herrn Martin Hunziker(Leiter der IT der KZO) verwiesen. Dieser nahm uns sehr gerne mit der Anmerkung auf, dass die Umsetzung unserer Idee sehr schwierig sein werde.
        %Letzteres, also Unterstützung, war bei uns sehr schwierig, denn keine Lehrperson kennt sich mit Unity und C\# genug gut aus.
        
    \item \textbf{Plagiat} \\
        Arbeiten dürfen nicht einfach kopiert werden. \\Wir haben uns zwar bei gewissen Problemen online inspirieren lassen, jedoch sind die verwendeten Lösungen keine Plagiate.
        Unser gesamter Code sowie unsere schriftliche Arbeit sind von uns selbst geschrieben.
\end{enumerate}

\section{Anforderungen an das Spiel}
\label{chap:foo}
Gewisse Features sind für das Funktionieren des Spiels wichtiger als andere.
Zum Beispiel sind für uns Einstellungen, Truppen und Multiplayer notwendiger als die Monetarisierung.
Deshalb ist der folgende Abschnitt in vier Gruppen unterteilt:
\begin{enumerate}
    \item \textbf{Notwendig:}
        Ohne diese Features läuft das Spiel nicht oder ist sehr mangelhaft,
        voller Fehler und unspielbar. Ohne diese Funktionen ist Idee des Spieles nicht erkennbar.
    \item \textbf{Anvisiert:}
        Diese Features sind unser Ziel. Sie wurden Herrn Hunziker angekündigt und sind alle in unserem Plan festgehalten.
        Diese Ziele sollen dazu führen, dass das Spiel
        gut spielbar ist und es Spass macht. Sie sind auch dafür essenziell, das Spiel nicht zu monoton zu gestalten und sollten
        alle bis zum Abgabetermin der schriftlichen Arbeit vollendet sein.
    \item \textbf{Möglichkeiten:}
        Dies sind Möglichkeiten das Spiel noch auf die Spitze zu bringen. Keine dieser Anforderungen ist für das Spielen nötig,
        jedoch können sie das Spielerlebnis spannender und ausgereifter machen. Diese
        Ideen sind entweder in den Sternen geschrieben oder noch bis zur mündlichen Präsentation zu erreichen.
        Vereinzelt können diese, wie die anvisierten Anforderungen, noch bis zur schriftlichen Abgabe erreicht werden.
    \item \textbf{Verworfen:}
        Weitere Ideen, welche wir hatten, die wir jedoch als unmöglich oder nicht sinnvoll erachten.
\end{enumerate}

\subsection{Notwendig}
\begin{itemize}
    \item \textbf{Systemanforderungen} \\
        Unser Spiel soll auf den meisten modernen Computern funktionieren. Es soll auf macOS und
        Windows fliessend gespielt werden können(mindestens 60FPS).
    \item \textbf{Benutzeroberfläche} \\
        Die Startseite soll Folgendes enthalten: einen ''Spielen''-Knopf, einen ''Einstellungen''-Knopf, einen
        ''Credits''-Knopf und einen ''Verlassen''-Knopf. Die Benutzeroberfläche soll intuitiv bedienbar sein und
        schön aussehen. Sie soll mit der Maus bedienbar sein, andere Eingabemöglichkeiten
        werden nicht unterstützt.
    \item \textbf{Lokaler Multiplayer} \\
        Das Spiel soll im lokalen Netz gespielt werden können, so zum Beispiel mit Freunden oder Familie.
        Es sollte mithilfe der IP-Adresse innerhalb des gleichen Netzes zusammengespielt werden können.
    \item \textbf{Einstellungen} \\
        Die Auflösung, Vollbild und Lautstärke müssen im UI eingestellt werden können.
    \item \textbf{Kamera} \\
        Die Kamera ist beweglich und das ganze Schlachtfeld ist sichtbar.
    \item \textbf{Truppen} \\
        Es braucht Truppen, die für einen Spieler kämpfen können. Diese sind für den Spielverlauf notwendig.
        Und ohne sie hat das Spiel keinen Sinn.
    \item \textbf{Gewinnmöglichkeit} \\
        Es soll eine Möglichkeit geben, das Spiel zu gewinnen oder zu verlieren und es somit zu beenden. Diese Anforderung kann sehr simpel
        mit einer Linie verwirklicht werden, welche beim Überschreiten einer Truppe das Spiel beendet.
    \item \textbf{Lanes} \\
        Unser Spiel ist zwar 2.5D, aber hat dennoch eine Tiefe.
    \item \textbf{Synchronisation} \\
        Unser Spiel soll in Echtzeit spielbar sein, deswegen müssen die beiden Spieler synchronisiert dasselbe sehen. Eine De-Synchronisation wäre verheerend. 
    \item \textbf{Crossplay}
        Wir wollen die Möglichkeit erreichen, sich mit einem Windows Rechner und einem macOS Rechner zu verbinden und zusammenzuspielen.
\end{itemize}

\subsection{Anvisiert}
\begin{itemize}
    \item \textbf{Helden} \\
        Beide Spieler haben eine Art Truppe, welche ihre Siegeslinie beschützt. Sie haben einen Racheeffekt, welcher die
        Lane ausradiert, damit das Spiel nicht sofort vorbei ist.
    \item \textbf{Design} \\
        Das Spiel sollte vom Aussehen her "was hergeben". Es sollte nicht wie ein Prototyp, sondern wie ein 
        vollendetes Spiel aussehen. 
    \item \textbf{Deck} \\
        Zu Beginn können die Spieler ihr eigenes Deck aus einer Auswahl von Karten zusammenstellen. Im Spiel werden aus diesem Kartendeck per Zufall Karten gezogen.
    \item \textbf{Bot} \\
        Dies ist ein sehr simpler Algorithmus, um alleine gegen den Computer zu spielen. Die Schwierigkeitsstufen kann zwischen \textit{'Einfach', 'Mittel', 'Schwierig'} gewählt werden.
        Angepasst an den Schwierigkeitsgrad werden zufällig Truppen geschickt werden.
    \item \textbf{Truppen}\\
    Es gibt verschiedene Truppenarten:
    \begin{itemize}
        \item \textbf{Nahkampftruppe:}
            wenig Reichweite
        \item \textbf{Fernkampftruppe:}
            Greift auf grosse Reichweite an. Sie schiesst Projektile(z.B. ein Bogenschütze, der mit Pfeilen schiesst).
        \item \textbf{Suizidtruppe}
            Stirbt bei Berührung mit einem Gegner und löst einen Effekt aus.
    \end{itemize}
    \item \textbf{Effekte}
    \begin{itemize}
        \item \textbf{Gift:}
            Zeitlich limitierter und wiederholender Schadenseffekt. \\Eine visuelle Markierung soll vorhanden sein
            und der Gifteffekt darf sich nicht kumulieren.
        \item \textbf{Dornen:}
            Truppen, welche Charaktere mit Dornen attackieren, erhalten Schaden. Der Effekt soll nach Auswahl auf Nahkämpfer,
            Fernkämpfer oder beides wirken.
        \item \textbf{Rache:}
            Truppen mit dem Effekt Rache haben einen Effekt nach ihrem Tod. Zum Beispiel kann durch diesen Tod eine Truppe beschwört oder Schaden verursacht werden.
    \end{itemize}
\end{itemize}

\subsection{Möglichkeiten}
\begin{itemize}
    \item \textbf{Zauber} \\
        Alternativ zum Herbeirufen einer Truppen können gewisse Karten einen Effekt auslösen(Zauber).\\
        Beispiele für Zauber:
            \begin{itemize}
                \item ''Heile deine Helden um 5 Leben''
                \item ''Gib deinen Helden 5 Rüstungspunkte''
                \item ''Verursache allen gegnerischen Truppen 5 Schadenspunkte''
                \item ''Ziehe 3 Karten''
            \end{itemize}
    \item \textbf{Helden} \\
        Eine Auswahl von Helden, mit unterschiedlichem Schaden. \\Leben und Effekte des jeweiligen Helden, würde das Spiel nochmals spannender machen. Auch sollten gewisse Truppen auf bestimmte Helden limitiert sein. 
    \item \textbf{Truppen}
        Weitere Truppen können jederzeit erstellt werden. Hier ist kein Limit gesetzt. Leben, Schaden, Darstellung, Effekt und vieles mehr kann angepasst werden.
    \item \textbf{Effekte}
    \begin{itemize}
        \item \textbf{Wiederbelebung:}
            Die Truppe sofort an Ort und Stelle oder mit einer Verzögerung am Startpunkt wiederbelebt.
        \item \textbf{Rüstung:}
            Die Truppe hat zusätzliche Rüstungspunkte zu den Lebenspunkten. Die Rüstung wird zuerst abgezogen und hat im Gegensatz zum Leben keine Limite.
        \item \textbf{Aura:}
            Die Truppe fügt gegnerischen Truppen in einem gewissen Radius permanent Schaden zu.
    \end{itemize}
    \item \textbf{Design} \\
        Wenn ausreichend Zeit vorhanden ist, kann das Design immer weiter verbessert werden. Hier geht es um den Feinschliff, z.B. mehr Hintergründe, Details und mehr selbsterstellte Design.
    \item \textbf{Tutorial} \\
        Eine Anleitung für Anfänger wäre sehr schön. Sie soll neuen Spielern das Anfangen erleichtern. Es sollte aber auch überspringbar sein,
        damit erfahrene Spieler, es nicht erneut spielen müssen. Es soll weder allzu lang noch zu schwierig sein. Es soll am Besten "Anhand von Gameplay" alle Mechaniken des Spiels
        erklären.
    \item \textbf{Mehrsprachig} \\
        Das Spiel soll in Englisch, Deutsch und Französisch spielbar sein.
    \item \textbf{Ingame-Chatfenster}\\
        Ein Textfeld, in dem man sich mit dem Gegner unterhalten kann, würde einen zusätzlichen "Fun-Effect" bieten.
    \item \textbf{Monetarisierung} \\
    Es sind uns drei Möglichkeiten eingefallen, wie wir mit dem Spiel Geld verdienen könnten. Es folgen jeweils die Vor- und Nachteile:
    \begin{enumerate}
        \item \textbf{Kaufbare Gegenstände}
        \begin{itemize}
            \item[+] Vielseitige und nachhaltige Monetarisierung. Neue Features bedeuten auch neue Einnahmemöglichkeiten. Die käuflichen Inhalte sollten
                        auch erspielbar sein, somit kann bezahlen zwar zu Vorteilen führen, jedoch mit einer hohen Spielzeit trotzdem erreichbar sein.
            \item[-] Pay-to-Win
        \end{itemize}
        \item \textbf{Skins}
        \begin{itemize}
            \item[+] Weit verbreitet und sehr beliebt bei Spielern. Gibt keinem Spieler Vorteile.
            \item[-] Wenig Nutzen und sehr aufwändig. Neue Designs zu erstellen, wäre bei uns nicht so sinnvoll.
                        Wir möchten noch sehr viel Features implementieren und benötigen dafür viel Zeit. Zudem sind keine spezialisierten Designer. Der Aufwand für uns eine ansprechende und brauchbare Darstellung zu erschaffen steht in keinem sinnvollen Verhältnis zum Mehrwert für das Spiel.
        \end{itemize}
        \item \textbf{Kostenpflichtiges Spiel}
        \begin{itemize}
            \item[+] Einmalige Bezahlung, was für viele Nutzer lukrativer ist. Jedoch gilt dies vor allem bei Singleplayer Spielen.
            \item[-] Wir nehmen an, dass wenige Nutzer bereit wären, Geld für unser Spiel zu bezahlen.
                    Dafür wird es vorerst nicht genug ausgereift sein.
                    Auch ist diese Methodik nicht nachhaltig und führt nur zu einer einmaligen Geldspritze.
                    Viele grössere Spiele führen deshalb später
                    DLCs ein, um das Spiel zu erweitern.
                    Jedoch ist dies bei einem Multiplayer Spiel Pay-to-Win.
                    Wir müssen mit hoher Wahrscheinlichkeit vorest Geld investieren, um unser Spiel anbieten zu können.
        \end{itemize}
    \end{enumerate}
\end{itemize}

\subsection{Bereits früh verworfen}
\begin{itemize}
    \item \textbf{Online Multiplayer} \\
        Besser als ein Multiplayer, der limitiert auf dasselbe Netz ist, ist ein Multiplayer, der weltweit verfügbar ist. Jedoch ist die direkte Verbindung zwischen zwei Computern, die sich in unterschiedlichen Netzen befinden, dank der heutigen Firewalls von Routern und Computern, für uns nahezu unmöglich. 
        Es würden sich viele weitere Probleme ergeben, wie zum Beispiel Port-Forwarding. Dies ist einer der Gründe, weshalb ein Server sehr praktisch wäre. Mit diesem wäre dieses
        Problem gelöst. Server sind aber teuer und aufwändig zu programmieren und zu warten. 
    \item \textbf{Shop} \\
        Nicht alle Karten sind von Beginn an freigegeben, sondern müssen freigespielt werden.
        So kann zum Beispiel eine Ingame Währung erspielt werden und damit im Shop Karten oder sonstige Dinge gekauft werden.
        Der Shop kann mit Zahlungsmethoden ausgestattet werden und so zur eventuellen Monetarisierung beitragen.
    \item \textbf{Kampagne} \\
        Im Singleplayer spielt man gegen bestimmte vorprogrammierte Gegner. Sie sollen immer stärker werden und bestimmte Herausforderungen mit sich bringen.
        Bei einem Sieg werden Belohnungen verteilt.
    \item \textbf{Anti-Cheat} \\
        In allen grösseren Spielen ist eine sehr komplexe Software vorhandeln, um das Schummeln zu verhindern.
        Teilweise geschieht dies auf Systemebene, oder sogar auf Kernel-Level.
    \item \textbf{Errungenschaften} \\
        Durch Erfüllung von Aufgaben erhält man Belohnungen. 
        Eine Übersicht und die Möglichkeit die Errungenschaften zu erfüllen.
        Gegebenenfalls Belohnungen verteilen, wie zum Beispiel bestimmte Karten freischalten.
        Diese Erweiterung ist keines Falls notwendig und ein absolutes nice-to-have Feature.
\end{itemize} 

% \includegraphics[height=10cm, width=\textwidth]{resources/mockups/mockup-login}