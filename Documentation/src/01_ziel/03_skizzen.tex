\chapter{Ideen}

\section{Skizzen}

\includegraphics*[width=7cm]{resources/SK_startpage.png} \quad \includegraphics*[width=7cm]{resources/SK_auswahl.png}\\
\textit{Startseite} \qquad \qquad \qquad \qquad \qquad \qquad \qquad \quad \textit{Auswählen von Karten und Helden}
\\
\begin{center}
    \includegraphics*[width=14.5cm]{resources/sk_gamemain.png}\\
\end{center}
\textit{Spielszene mit UI}

\begin{center}
    \includegraphics*[width=12cm]{resources/sk_dragndrop.png}
\end{center}
\qquad \quad \enspace \textit{Mechanik der Karten / Spawnen von Truppen}

\section{Mindmap}
\begin{center}
    \includegraphics*[width=14.5cm]{resources/sk_mindmap1.png}
\end{center}


\section{Überlegungen zum Spielprinzip}

Die Idee des Spielprinzips wird im folgenden Unterkapitel grob beschrieben.\
\begin{itemize}
    \item In unserem Spiel treten zwei Spieler gegeneinander an. Dabei befinden sich die Startbereiche der Spieler auf der linken bzw. rechten Seite einer steinernen Brücke.
    \item In die Schlacht begleitet wird jeder Spieler von vier Helden. Diese Helden schreiten zu Beginn des Spieles durch die magischen Portale und bilden eine erste Verteidigungslinie.
    Sterben jene Helden, lösen sie einen Zauber aus, der alle gegnerischen Truppen auf dieser Breite der Brücke ausradiert.
    \item Der Spieler kann zudem Unterstützung aus dem eigenen Lager befehligen. Die bereiten Truppen erhält der Spieler in Form von Karten. Der Spieler kann mit der Verschiebung dieser Karten auswählen, durch welches Portal sie schreiten sollen.
    \item Jede Truppe verlangt eine unterschiedliche Menge an Kraft, um das Portal zu durchschreiten. Die zur Verfügung stehende Mana (spirituelle Energie) wird oben recht angezeigt.
    \item Nach jeder Durchschreitung muss der Spieler eine bestimmte Zeit lang warten, bis er die nächste Truppe auswählen kann.
    \item Gelingt es einer verbündeten Einheit das gegnerische Portal zu durchschreiten, ist das Spiel vorbei und man hat gewonnen. 
    \item Mit den Pfeiltasten kann man steuern, welchen Teil des Schlachtfelds man überblicken will.
\end{itemize}



% \includegraphics[width=\textwidth]{resources/diagrams/domain-model}
