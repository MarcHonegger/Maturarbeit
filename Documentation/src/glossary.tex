\newglossaryentry{feature}
{
  name={Feature},
  description={Eine bestimmte Funktion, welche die Software weiterbringt}
}

\newglossaryentry{online}
{
  name={Online},
  description={nicht nur lokal auf dem eigenen Rechner, sondern im Internet}
}

\newglossaryentry{lokal}
{
  name={Lokal},
  description={Im selben Netz oder Computer}
}

\newglossaryentry{anti-Cheat}
{
  name={Anti-Cheat},
  description={Software um Schummeln zu verhindern.
  Ist sehr komplex und bei allen grösseren online Videospielen vorhanden.
  Teilweise sehr tief in das System vernetzt}
}

\newglossaryentry{cheater}
{
  name={cheater},
  description={Jemand, der unerlaubte Methoden oder Programme verwendet, um sich einen unfairen Vorteil zu verschaffen.
  Auf Deutsch \textit{ein Schummler}}
}

\newglossaryentry{Skins}
{
  name={Skins},
  description={Kaufbare Darstellungen, welche von dem Standard abweichen.
  In vielen Spielen nur im Tausch mit echter Währung zu erhalten und bringen nur ästhetische Unterschiede}
}

\newglossaryentry{Ingame}
{
  name={Ingame},
  description={Im laufendem Spiel, nicht auf dem Computer oder beim Start oder Schliessen des Spiels}
}

\newglossaryentry{pay-to-win}
{
  name={Pay-to-Win},
  description={Übersetzt: Bezahlen-zum-Gewinnen, bedeutet, dass mit echter Währung Ingame Vorteile erkauft werden können}
}

\newglossaryentry{firewall}
{
  name={Firewall},
  description={Schutzsystem eines technischen Systems um Viren fernzuhalten.
  Blockiert aus Sicherheit manchmal mehr als nötig und erschwert so zum Beispiel das Verbinden von zwei Computern über das Internet}
}

\newglossaryentry{shop}
{
  name={Shop},
  description={ein Ort im Spiel, in dem Dinge gekauft werden können. Ein Ingame-Laden}
}


\newglossaryentry{multiplayer}
{
  name={Multiplayer},
  description={Mehrspieler, also mehrere Spieler spielen zusammen das gleiche Spiel.}
}

\newglossaryentry{singleplayer}
{
  name={Singleplayer},
  description={Einzelspieler, das Spiel wird alleine, meist nur lokal, gepielt}
}

\newglossaryentry{tutorial}
{
  name={Tutorial},
  description={Eine Anleitung, in Videospielen meist eine kurze Spielsequenz, welche Tipps und eine fixe Abfolge besitzt}
}

\newglossaryentry{helden}
{
  name={Helden},
  description={In Videospielen der wichtigste Charakter.
  In gewissen Spielen ist der Held der Charakter, den man spielt.
  Der Tod des Helden bedeutet einen massiven Nachteil}
}

\newglossaryentry{kamera}
{
  name={Kamera},
  description={Die Kamera nimmt in Unity auf, was der Spieler sieht.
  Mit Kamera ist also das gemeint, was der Spieler sieht.}
}

\newglossaryentry{lane}
{
  name={Lane},
  description={
    Bezeichnet generell eine Linie oder Bahn auf der gegangen werden kann.
    Sehr prägnant in MOBAs wie League of Legends oder Dota.
    In unserem Spiel eine Bahn der vier, auf welcher Truppen beschwört werden können.}
}

\newglossaryentry{steam}
{
  name={Steam},
  description={
    Steam ist der grösste Videospiel-Anbieter der Welt.
    Gehört dem Unternehmen Valve.}
}

\newglossaryentry{indie developer}
{
  name={India Developer},
  description={
    Ein Indie Developer entwickelt Indie Games, kurz für "Independent Game".
    Diese werden von einer einzelnen Person oder kleinen Game Studios entwickelt.
    Der Gegensatz sind AAA-Spiele.
  }
}

\newglossaryentry{aaa-spiel}
{
  name={AAA-Spiel},
  description={
    Ausgesprochen "Tripple A", sind dies Videospiele mit maximalem Budget und meist erstellt von den grössten Gamestudios.
  }
}

\newglossaryentry{game studio}
{
  name={Game Studio},
  description={
    Ein Unternehmen, dass Videospiele entwickelt.
  }
}



% \newglossaryentry{slalom}{
%     name={Slalom},
%     description={A kind of \gls{contract} in the \gls{coiffeur-jass} which plays without a trump.
%     Alternates between \gls{bottoms-up} and \gls{tops-down} or vice-versa, scoring rules apply from the first chosen option}
% }
