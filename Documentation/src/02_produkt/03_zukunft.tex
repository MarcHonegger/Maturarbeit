\chapter{Zukunft}
\textit{''Der beste Weg, die Zukunft vorauszusagen, ist, sie selbst zu gestalten.''(Abraham Lincoln)}

\section{Ein weisses Blatt}
\begin{itemize}
    \item[-] Das Spiel wird bis zu der Präsentation nicht nur ''Staub ansetzen''. Wir haben fest vor (sofern es der Prüfungsdruck der kommenden Monaten erlaubt), das Spiel noch weiterzuentwickeln und zu verbessern. Denn trotz grossem Fortschritt
    ist Lanceclash noch lange nicht fertig. Obwohl schon sehr viel existiert, haben wir noch sehr viele Ideen, die wir gerne umsetzen möchten.
    \item[-] Viele Ideen wurden aus Zeitgründen fürs Erste verworfen oder verzögert. Daher mangelt es uns auf keinen Fall an Ideen. Wir wollen diese
    fehlenden Features auf jeden Fall noch hinzufügen.
    \item[-] Das Hinzufügen der meisten Ideen sollte sich dabei als nicht sehr schwierig herausstellen. So existiert zum Beispiel die Mechanik für den Effekt ''Dornen'' bereits. 
    Sie benötigt lediglich noch einen Feinschliff. 
    \item[-] Hätten wir unendlich viel Zeit, würden wir einen kompletten Recode des ganzen Spiels schreiben. Im Nachhinein wissen wir, dass wir viele Stellen besser hätten schreiben können. 
    Zudem hätten wir den Vorteil, dass wir bereits auf die ungefähre Struktur zurückgreifen können. 
\end{itemize}

\section{Alpha- und Closed Beta-Testing}
\begin{itemize}
    \item[-] Bis zum Abschluss der Arbeit wurde ein Alpha-Testing kleinen Testgruppe durchgeführt. Bis zu der Präsentation wird
    das Alpha-Testing ausgeweitet und ein geschlossenes Beta-Testing ist bereits in Planung. 
    \item[-] Das Beta-Testing war ab der ersten Sommerferienwoche geplant. Allerdings waren wir zu diesem Zeitpunkt noch nicht zufrieden mit unserem Produkt und hatten auch noch kein Alpha-Testing durchgeführt. Wir wollten
    auf keinen Fall etwas unserer Meinung nach Ungenügendes aus der Hand geben. So wurde das Datum des Beta-Testings immer weiter nach Hinten verschoben, bis es vor dem Abgabetermin der schriftlichen Arbeit nicht mehr realisierbar war.
    \item[-] An Testern wird es auf keinen Fall mangeln. Alle Personen in unserem Umfeld wollen
    das Spiel ausprobieren. Ob Schulkollegen, Freunde oder Verwandte; sie alle wollen unser Spiel in den Händen halten. Wir sind froh, in einem solchen Umfeld arbeiten zu können.
    \item[-] Wir hatten bereits die Idee, dass sich mögliche Tester auf einer Warteliste eintragen können. Wir hatte sogar angedacht, einen kleinen Server zu mieten,
    um eine kleine Webseite hochzuschalten auf der sich die Testinteressierten eintragen könnten. Die Beta-Tester würden dann per Los ausgewählt werden.
\end{itemize}

\section{Veröffentlichung}
\begin{itemize}
    \item[-] Wir planen die erste vollständige Version für die Öffentlichkeit zugänglich zu machen. V 1.0 wird entweder auf itch.io (\url{https://itch.io/}) oder auf Steam 
    (\url{https://store.steampowered.com/}) veröffentlicht. Welcher Preis dafür festgelegt wird, ist noch unklar.
    \item[-] Vielleicht werden wir auch nur einen ''Buy Me A Coffee''-Button (\url{https://www.buymeacoffee.com/}) hinzufügen, über den man uns unterstützen kann. 
    \item[-] Der Source-Code bleibt allerdings ziemlich sicher privat. Es ist möglich, dass wir noch darüber diskutieren werden, ob dies auch so bleiben wird.
\end{itemize}



