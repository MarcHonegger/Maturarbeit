\chapter{Zukunft}
\textit{''Der beste Weg, die Zukunft vorauszusagen, ist, sie selbst zu gestalten.''(Abraham Lincoln)}

\section{Ein weisses Blatt}
\begin{itemize}
    \item[-] Das Spiel wird bis zu der Präsentation nicht nur Staub ansetzen. Wir haben fest vor, das Spiel noch weiterzuentwickeln und zu verbessern. Denn trotz grossem Fortschritt
    ist das Spiel noch lange nicht fertig. Obwohl schon sehr viel existiert, haben wir noch sehr viele Ideen, die wir gerne umsetzen möchten.
    \item[-] Viele Ideen wurden aus Zeitgründen fürs Erste verworfen oder verzögert. Dies bedeutet allerdings, dass es uns auf keinen Fall an Ideen mangelt. Wir wollen diese
    fehlenden Features auf jeden Fall noch hinzufügen.
    \item[-] Das Hinzufügen der meisten Ideen sollte sich dabei nicht sehr schwierig herausstellen. So existiert zum Beispiel die Mechanik für den Effekt ''Dornen'' bereits. 
    Es benötigt lediglich noch einen Feinschliff. 
    \item[-] Hätten wir unendlich viel Zeit, würden wir einen kompletten Recode des ganzen Spiels tätigen. Im Nachhinein wissen wir, dass wir viele Stellen besser schreiben hätten können. 
    Zudem hätten wir dann den Vorteil, dass wir bereits ungefähr wissen, wie alles aufgebaut sein soll. 
\end{itemize}

\section{Alpha- und Closed Beta-Testing}
\begin{itemize}
    \item[-] Geplant war ein intensives Testing ab der ersten Woche der Sommerferien. Allerdings waren wir zu diesem Zeitpunkt gar nicht zufrieden mit unserem Produkt. Wir wollten
    auf keinen Fall etwas unserer Meinung nach Ungenügendes aus der Hand geben. So wurde das Datum des Testing immer weiter verschoben, bis es gar kein Datum mehr gab.
    \item[-] Dies soll sich in der Zukunft allerdings ändern. Es gab zwar ein geschlossenes Alpha-Testing, allerdings war dies nur minimal vorhanden. Bis zu der Präsentation wird
    das Alpha-Testing ausgeweitet und ein geschlossenes Beta-Testing ist bereits in Planung. 
    \item[-] An Testern wird es auf keinen Fall mangeln. Alle Personen in unserem Umfeld wollen
    das Spiel ausprobieren. Ob Schulkollegen, Freunde oder Verwandte; sie alle wollen unser Spiel in den Händen halten. Wir sind froh, in einem solchen Umfeld arbeiten zu können.
    \item[-] Wir hatten bereits die Idee, dass sich mögliche Tester auf einer Warteliste eintragen können. Dafür wurde bereits besprochen, ob wir einen kleinen Server mieten wollen,
    um eine kleine Webseite hochzuschalten. Die Beta-Tester würden dann per Los ausgewählt werden.
\end{itemize}


\section{Veröffentlichung}
\begin{itemize}
    \item[-] Wir planen die erste vollständige Version für die Öffentlichkeit zugänglich zu machen. V 1.0 wird entweder auf itch.io (\url{https://itch.io/}) oder auf Steam 
    (\url{https://store.steampowered.com/}) veröffentlicht. Welchen Preis dafür festgelegt wird, steht noch in den Sternen geschrieben.
    \item[-] Vielleicht werden wir auch nur einen ''Buy Me A Coffee''-Button (\url{https://www.buymeacoffee.com/}) hinzufügen, über den man uns unterstützen kann. 
    \item[-] Der Source-Code bleibt allerdings ziemlich sicher privat. Es ist allerdings möglich, dass wir noch darüber diskutieren werden. 
\end{itemize}



